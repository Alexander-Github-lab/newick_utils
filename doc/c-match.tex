%&context

\section[sct_match]{Finding subtrees in other trees}


\match{} tries to match a (typically smaller) "pattern" tree to one or more
"target" tree(s). If the pattern matches the target, the target tree is
printed. Intuitively, a pattern matches a target if one can superimpose it onto
the target without "breaking" either. More accurately, the following happens
(in both trees):
\startitemize[n]
	\item leaves with labels found in both trees are kept, the other ones are
		pruned
	\item inner labels are discarded
	\item both trees are ordered (as done by \order{}, see  \in{}[sct_order])
	\item branch lengths are discarded
\stopitemize
At this point, the modified pattern tree is compared to the modified target, and if the \nw{} strings are identical, the match is successful.

\subsubsection{Example: finding trees with a specified  subtree topology}

File \filename{hominoidea.nw} contains seven trees corresponding to successive
theories about the phylogeny of apes (these were taken from
\from[URL:Hominoidea]). Let us see which of them group
humans and chimpanzees as a sister clade of gorillas (which is the current
hypothesis).

\page[no]
Here are small images of each of the trees in \filename{hominoidea.nw}: \\

\startcombination[2*4]
{\externalfigure[homino_0][scale=700]} {1 (until 1960)}
{\externalfigure[homino_1][scale=700]} {2 (Goodman, 1964)}
{\externalfigure[homino_2][scale=700]} {3 (gibbons as outgroup)}
{\externalfigure[homino_3][scale=700]} {4 (Goodman, 1974: orangs as outgroup)}
{\externalfigure[homino_4][scale=700]} {5 (resolving trichotomy)}
{\externalfigure[homino_5][scale=700]} {6 (Goodman, 1990: gorillas as outgroup)}
{\externalfigure[homino_6][scale=700]} {7 (split of {\em Hylobates})}
\stopcombination

Trees \#6 and \#7 match our criterion, the rest do not. To look for matching trees in \filename{hominoidea.nw}, we pass the pattern on the command line:

\typefile{match_1_txt.cmd}
\page[no]
\typefile{match_1_txt.out}


Note that only the pattern tree's topology matters: we would get the
same results with pattern \code{((Homo,Pan),Gorilla);},
\code{((Pan,Homo),Gorilla);}, etc., but not with
\code{((Gorilla,Pan),Homo);} (which would select trees \#1, 2, 3, and 5. In
future versions I might add an option for strict matching.

The behaviour of \match{} can be reversed by passing option \code{-v} (like
\code{grep -v}): it will print trees that {\em do not} match the pattern.
Finally, note that \match{} only works on leaf labels (for now), and assumes
that labels are unique in both the pattern and the target tree.
