\chapter{Introduction}

\externalfigure[star.pdf]

The \nutils{} are a set of \unix{} (including Mac OS X) and \unix-like (Cygwin) shell programs for working with phylogenetic trees. Their main features are:
\startitemize
 \item they require no user interaction\footnote{Why this is a good thing is not the focus of this document: I shall assume that if you are reading this, you already know when a command-line interface is better than an interactive interface.}
 \item they can work on any number of trees at a time
 \item they perform well with large trees
 \item they are implemented as filters\footnote{In \unix{} jargon, a
 \newconcept{filter} is a program that reads input from standard input and writes
 output to standard output.}
 \item they read and write text
\stopitemize
They are not tools for {\em making} phylogenies. Rather, they are for
processing existing ones, by which I mean manipulating the tree or extracting
information from it: rerooting, simplifying, extracting subtrees, printing
branch lengths and distances, etc - see \in{table}[tbl_prog_list]; a glance
at the table of contents should also give you an idea.

Each of the programs performs one task (with some variants). For example, here
is how you would reroot a series of phylograms contained in file
\code{mytrees.nw} using node \code{Dmelano} as the outgroup:

\starttyping
$ nw_reroot mytrees.nw Dmelano
\stoptyping 
Now, you might want to make cladograms from the rerooted trees. Program
\topology{} does the job, and since the utilities are filters, you can do it
all in a single command:
\starttyping
$ nw_reroot mytrees.nw Dmelano | nw_topology -
\stoptyping
As you can see, it is straightforward to pipe \nutils{} together, and of course they can be mixed freely with any other shell program (see e.g. \in{}[sct_counting_leaves]).

\subsection{About This Document}

This tutorial is organized as follows: chapter \in{}[chap_general] discusses
common features of the \nutils, chapter \in{}[chap_simple] shows examples of
simple tasks, and chapter \in{}[chap_adv] has examples of more advanced tasks. 

It is not necessary to read this material in order: you can pretty much jump to
any section in chapters \in{}[chap_simple] and \in{}[chap_adv], they do not
require reading previous sections. I would suggest reading chapter
\in{}[chap_general], then \in{section}[sct_display] because it explains how all
the tree graphics were produced.

The files for all the examples in this tutorial can be found in
subdirectory \code{data}.

All the program outputs are generated on-the-fly just before the document is run
through \ConTeXt{}, so they represent the actual output of the latest version of
the utilities.

\placetable[here][tbl_prog_list]{The \nutils{} and their functions}
\starttabulate[|l|l|]
\NC {\bf Program} \NC {\bf Function } \MR
\HL
\NC \clade		\NC	Extracts subtrees specified by node labels \NC\MR
\NC \condense	\NC	Condenses (simplifies) trees \MR
\NC \display	\NC	Shows trees as graphs (\ascii{} graphics or \svg) \MR
\NC \duration	\NC	Convert node ages into duration \MR
\NC \distance	\NC	Prints distances between nodes, in various ways \MR
\NC \ed				\NC	Stream editor (\foreign{\`{a} la} \code{sed} or \code{awk}); see also \luaed{} and \sched \MR
\NC \gen			\NC	Random tree generator \MR
\NC \nwindent	\NC	Shows Newick in indented form \MR 
\NC \labels		\NC	Prints node labels \MR
\NC \luaed				\NC	Like \ed, but uses Lua\MR
\NC \match		\NC	Finds matches of a tree in another one \MR
\NC \order		\NC	Orders tree (preserving topology) \MR
\NC \prune		\NC	Removes branches based on labels \MR 
\NC \rename		\NC	Changes node labels according to a mapping \MR
\NC \reroot		\NC	(Re)roots the tree \MR
\NC \sched				\NC	Like \luaed, but uses Scheme\MR
\NC \stats		\NC	Prints tree statistics and properties \MR
\NC \support	\NC	Computes bootstrap support of a tree given replicate trees \MR
\NC \topology	\NC	Alters branch properties, preserving topology \MR
\NC \trim		\NC	Trims a tree at a specified depth \MR
\stoptabulate

\subsection{Citing the \nutils}

If you use the \nutils{} for published work, please cite \nocite[Junier_2010]:
\placepublications[criterium=text]


