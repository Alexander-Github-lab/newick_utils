\chapter{Newick order}
\label{newick_order}

There are many ways of visiting a tree. One can start from the root and
proceed to the leaves, or the other way around. One can visit a node before
its children (if any), or after them. Unless specified otherwise, the
\nutils{} process trees as they appear in the \nw{} data. That is, for
tree \texttt{(A,(B,C)d)e;} the order will be A, B, C, d, e.

This means that a child always comes before its parent, and in particular,
that the root comes last. This is known as reverse post-order traversal, but
we'll just call it "\nw{} order".
