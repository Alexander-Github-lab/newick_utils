
\section{Rooting and Rerooting}
\label{sct_reroot}

Rooting transforms an unrooted tree into a rooted one, and rerooting changes a rooted tree's root. Some tree-building methods produce rooted trees (e.g., \textsc{UPGMA}), others produce unrooted ones (neighbor-joining, maximum-likelihood). 

The Newick format is implicitly rooted, in the sense that there is a 'top' node from which all other nodes descend. Some applications regard a tree with a trifuraction at the top node as unrooted. 

Rooting a tree is usually done by specifying an \textit{outgroup}. In the simplest case, this is a single leaf. The root is then placed in such a way that one of its children is the outgroup, while the other child is the rest of the tree (sometimes known as the \textit{ingroup}). 

Consider the following primate tree, \texttt{simiiformes\_wrong}:

\includegraphics{reroot_1_svg.pdf}

\noindent{}It is wrong because \texttt{Cebus}, which is a New World monkey (capuchin), should be the sister group of all the rest (Old World monkeys and apes, technically Catarrhini), whereas it is shown as the sister group of the macaque-colobus family, Cercopithecidae. We can correct this by re-rooting the tree using \texttt{Cebus} as outgroup:
\verbatiminput{reroot_2_svg.cmd}
which produces:

\includegraphics{reroot_2_svg.pdf} \\

\noindent{}Note that the root is placed in the middle of the ingroup-outgroup
edge.

The outgroup does not need to be a single leaf. The following tree is
wrong for the same reason as the one before, except that is has three New World
monkey species instead of one, and they appear as a clade (Platyrrhini) in the
wrong place:

\includegraphics{reroot_3_svg.pdf} \\

\noindent{}We can correct this by specifying the New World monkey clade as outgroup:

\verbatiminput{reroot_4_svg.cmd}

\includegraphics{reroot_4_svg.pdf} \\

\noindent{}Note that I did not include all three New World monkeys, only \texttt{Cebus} and \texttt{Allouatta}. This is because it is always possible to define a clade using only two leaves. The result would be the same if I had included all three, though. You can use inner labels too, if there are any:
\begin{verbatim}
$ nw_reroot simiiformes_wrong_3og Platyrrhini
\end{verbatim}
will reroot in the same way (not shown). Beware that inner labels are often used for support values (as in bootstrapping), which are generally not useful for defining clades.
