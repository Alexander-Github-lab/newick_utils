
\chapter{General Remarks}
\label{chap_general}

The following applies to all programs in the \nutils{} package.

\section{Help}
\label{sect_help}

All programs print a help message if passed option \texttt{-h}:

\begin{samepage}
\begin{verbatim}
$ nw_indent -h
Indents the Newick, making structure more clear.

Synopsis
--------

nw_indent [-cht:] <newick trees filename|->

Input
-----

Argument is the name of a file that contains Newick trees, or '-' (in
which case trees are read from standard input).

Output
------

By default, prints the input tree, with each parenthesis and each leaf on a
line of its own, and indented a multiple of '  ' (two spaces) to reflect
structure. The default output is valid Newick.
[...]
\end{verbatim}
\end{samepage}
The help page describes the program's purpose, its input and output, and its options, in a format reminiscent of \unix{} manpages. It also shows a few examples. All examples can be tried out using files in the \texttt{data} directory.

\section{Input}
\label{sect_input}

Since the \nutils{} are for working with trees, it should not be a surprise that all programs take Newick as input. The Newick format is a file format for (phylogenetic) trees. It is one of the most widely used and uderstood formats.
A complete description can be found at \url{http://evolution.genetics.washington.edu/phylip/newicktree.html}.

The input tree(s) are always the first argument to the program (after any options). They may either be stored in a file, or piped on \stdin{}. In the latter case, the filename is replaced by a '\texttt{-}' (dash):

\begin{samepage}
\begin{verbatim}
$ nw_display mytrees.nw
\end{verbatim}
is the same as
\begin{verbatim}
$ cat mytrees.nw | nw_display -
\end{verbatim}
\end{samepage}
Of course the second form is only really useful when chaining several programs into pipelines.

\subsection{Multiple Trees in Input}

Most programs can work on any number of trees. That is, in the example above the file \texttt{mytrees.nw} may contain one or more trees, preferably one tree per line\footnote{It may also work if the trees are formatted differently, but the programs were only tested on input files with one tree per line.}. The task will be performed on each tree in the input. So if you need to reroot 1,000 trees on the same outgroup, you can do it all in a single step (see \ref{sct_reroot}).

\section{Output}
\label{sect_output}

The \nutils{} either modify trees and print the result, or print information about the trees. In the first case, the output is also Newick. In the second case, the format depends on the program, but it is always text (\textsc{ascii} graphics, \svg, numeric data, textual data, etc.).

