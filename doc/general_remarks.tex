
\chapter{General Remarks}
\label{chap_general}

The following applies to all programs in the \nutils{} package.

\section{Help}
\label{sct_help}

All programs print a help message if passed option \texttt{-h}. Here are the
first 20 lines of \nwindent{}'s help:

\verbatiminput{general_1_txt.cmd}
\begin{samepage}
\verbatiminput{general_1_txt.out}
\end{samepage}

The help page describes the program's purpose, its input and output, and its
options, in a format reminiscent of \unix{} manpages. It also shows a few
examples. All examples can be tried out using files in the \texttt{data}
directory.

\section{Input}
\label{sct_input}

Since the \nutils{} are for working with trees, it should be no surprise
that the main input is a file containing trees. The trees must be in
Newick format, which is one of the most widely used tree formats. Its
complete description can be found at
\url{http://evolution.genetics.washington.edu/phylip/newicktree.html}.

The input file is always the first argument to the program (after any options).
It may be a file stored in a filesystem, or \stdin{}. In the latter case, the
filename is replaced by a '\texttt{-}' (dash):

\begin{samepage}
\begin{verbatim}
$ nw_display mytrees.nw
\end{verbatim}
is the same as
\begin{verbatim}
$ cat mytrees.nw | nw_display -
\end{verbatim}
or
\begin{verbatim}
$ nw_display - < mytrees.nw
\end{verbatim}
\end{samepage}

\noindent{}Of course the second ("dashed") form is only really useful when chaining several programs into pipelines.

\subsection{Multiple Input Trees}

The input file can contain one or more trees. When there is more than one, I
prefer to have one tree per line, but this is not a requirement: they can be
separated by any amount of whitespace, including none at all. The task will be
performed\footnote{well, attempted\ldots} on each tree in the input. So if you
need to reroot 1,000 trees on the same outgroup, you can do it all in a single
step (see \ref{sct_reroot}). 

\section{Output}
\label{sct_output}

All output is printed on \stdout{} (warnings and error messages go to \stderr).
The output is either trees or information about trees. In the first case,
the trees are in \nw{} format, one per line. In the second case, the format
depends on the program, but it is always text (\textsc{ascii} graphics, \svg,
numeric data, textual data, etc.).

\section{Options}
\label{sct_options}

Options change program behaviour and/or allow extra arguments to be passed.
They are all passed on the command line, before the mandatory argument(s),
using a single letter preceded by a dash, in the usual \unix{} way. There are
no mandatory control files, although some tasks require additional files (\eg{}
\ref{sct_display_svg_css}). For example, we saw above that \display{} produces
graphs. By default the graph is \ascii{} graphics, but with option \texttt{-s},
the program produces \svg:
\begin{verbatim}
$ nw_display -s sometree.nw
\end{verbatim}
All options are described in the program's help page (see \ref{sct_help}).
