
\chapter{General Remarks}
\label{chap_general}

The following applies to all programs in the \nutils{} package.

\section{Help}
\label{sct_help}

All programs print a help message if passed option \texttt{-h}. Here are the
first 25 lines of \nwindent{}'s help, which you obtain by typig
\verb+nw_indent -h+:
\begin{samepage}
\verbatiminput{general_1_txt.out}
\end{samepage}
The help page describes the program's purpose, its input and output, and its
options, in a format reminiscent of \unix{} manpages. It also shows a few
examples. All examples can be tried out using files in the \texttt{data}
directory.

\section{Input}
\label{sct_input}

Since the \nutils{} are for working with trees, it should be no surprise
that the main input is a file containing Newick trees. The trees must be in
Newick format, which is one of the most widely used and uderstood formats. A
complete description can be found at
\url{http://evolution.genetics.washington.edu/phylip/newicktree.html}.

The input file is always the first argument to the program (after any options).
It may be a file stored in a:w filesystem, or \stdin{}. In the latter case, the
filename is replaced by a '\texttt{-}' (dash):
\begin{samepage}
\begin{verbatim}
$ nw_display mytrees.nw
\end{verbatim}
is the same as
\begin{verbatim}
$ cat mytrees.nw | nw_display -
\end{verbatim}
\end{samepage}
Of course the second form is only really useful when chaining several programs into pipelines.

\subsection{Multiple Trees in Input}

The input file can contain one or more trees, preferably one tree per
line\footnote{It may also work if the trees are formatted differently, but the
programs were only tested on input files with one tree per line.}. The task
will be performed on each tree in the input. So if you need to reroot 1,000
trees on the same outgroup, you can do it all in a single step (see
\ref{sct_reroot}). 

\section{Output}
\label{sct_output}

All output is printed on \stdout{} (error messages go to \stderr). The \nutils{}
either print trees or information about trees. In the first case, the output is
also Newick. In the second case, the format depends on the program, but it is
always text (\textsc{ascii} graphics, \svg, numeric data, textual data, etc.).

\section{Options}
\label{sct_options}

Options change program behaviour and/or allow extra arguments to be passed.
They are all passed on the command line, before the mandatory argument(s),
using a single letter preceded by a dash - in the usual \unix{} way. There are
no mandatory control files, although some tasks require additional files (\eg{}
\ref{sct_display_svg_css}). For example, we saw above that \display{} produces
graphs. By default the graph is \ascii{} graphics, but with option \texttt{-s},
the program produces \svg:
\begin{verbatim}
$ nw_display -s sometree.nw
\end{verbatim}
All options are described in the program's help page (see \ref{sct_help}).
