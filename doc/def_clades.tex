\chapter{Defining Clades by their Descendants}

When you need to specify a clade using the \nutils{}, you either give the label
of the clade's root, or the labels of (some of) its descendants. Since inner
node rarely have labels (or worse, have unuseable labels like bootstrap support
		values), you will often need to specify clades by their
descendants.

Consider the following tree:

\begin{center}
\includegraphics{reroot_1og_corr_svg.pdf} 
\end{center}

Suppose we want to extract the Hominoidea clade - the apes. It is the clade
that contains \texttt{Homo}, \texttt{Pan} (chimps), \texttt{Gorilla},
     \texttt{Pongo} (orangutan), and \texttt{Hylobates} (gibbons). The clade is
     not labeled in this tree, but this list of labels defines it without
     ambiguity. In fact, we can define it unambiguously using just
     \texttt{Hylobates} and \texttt{Homo} - or \texttt{Hylobates} and any other
     label. The point is that \emph{you never need more than two labels to
     unambiguously define a clade}.

You cannot choose any two nodes, however: the condition is that the last
common ancestor of the two nodes be the root of the desired clade.
