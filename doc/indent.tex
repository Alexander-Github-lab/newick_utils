
\section{Indenting}
\label{sct_indent}

\nwindent{} shows a \nw{} tree in indented form. This shows the structure more
clearly than the compact form. For example, this is a tree in compact form, in
file \texttt{falconiformes}:
\verbatiminput{falconiformes}
And this is the same tree, indented:
\verbatiminput{indent1_nw.cmd}
\verbatiminput{indent1_nw.out}
The structure is much more clear, it is also relatively easy to edit manually
in a text editor. An added benefit is that it is still valid \nw. 

Yet another advantage of indenting is that it is resistant to certain errors
which would cause \display{} to fail.\footnote{This is
because indenting is a purely lexical process, hence it does not need a
syntactically correct tree.} For example, there is an error in this tree:
\verbatiminput{falconiformes_error}
yet it is hard to spot, and trying \display{} won't help as it will generate a
parse error. With \nwindent{}, however, you can at least look at the tree:
\verbatiminput{indent4_nw.out}
While the error is not exactly obvious, you can at least view the Newick. It turns out there is a comma missing after \texttt{Sagittarius}.

The indentation can be varied by supplying a string (option \texttt{-t}) that
will be used instead of the default (which is two spaces). If you want a deeper
indentation, you could say this:
\verbatiminput{indent2_nw.cmd}
\verbatiminput{indent2_nw.out}
You can also use option \texttt{-t} to highlight indentation:
\verbatiminput{indent3_txt.cmd}
\verbatiminput{indent3_txt.out}
Now the indentation levels are easier to see, but at the expense of the tree no
longer being valid \nw.




