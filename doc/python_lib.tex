\lstset{
	language=Python,
	basicstyle=\ttfamily,
	keywordstyle=\color{NavyBlue},
	stringstyle=\color{OliveGreen},
	commentstyle=\color{red},
	frame=single,
	frameround=tttt,
	framexleftmargin=5mm,
	numbers=left
}

\chapter{Python Bindings}
\label{chap_python_lib}

This chapter shows how you can use the \nutils{}'s functionalities in Python
programs. while the \nutils{} are written in C, the \texttt{ctypes}
module\footnote{Available in Python 2.5 and up} makes it easy to access them
from python, and the distribution contains a module, \texttt{newick\_utils.py},
that provides an object-oriented interface to the underlying C code.

Let's say we want to add a utility that prints simple statistics about trees,
like the number of nodes, the depth, whether it is a cladogram or a phylogram,
etc. We will call it \texttt{nw\_info.py}, and for now we'll pass it a \nw{}
file on standard input, so the usage will be something like:

\begin{verbatim}
$ nw_info.py < data/catarrhini
\end{verbatim}

\noindent{}Our first program will just print the trees' type (cladogram or
phylogram).

\begin{lstlisting}
from newick_utils import *

for tree in Tree.parse_newick_input():
    print tree.get_type()
\end{lstlisting}

\noindent{}Line 1 imports definitions from the \texttt{newick\_utils.py}
module. Line 3 iterates over all the trees in input, returning an object of
class \texttt{Tree} for each of them. The type of this tree is printed on line
4.

This program is very simple, but more complicated utilities may have similar
structures: iterate over all the input trees, doing something with them.

\begin{lstlisting}
from newick_utils import *

for tree in Tree.parse_newick_input():
    # process tree
\end{lstlisting}
