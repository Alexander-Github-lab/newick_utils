\lstset{
	language=Python,
	basicstyle=\ttfamily,
	keywordstyle=\color{NavyBlue},
	stringstyle=\color{OliveGreen},
	commentstyle=\color{red},
	frame=single,
	frameround=tttt,
	framexleftmargin=8mm,
	numbers=left
}

\chapter{Python Bindings}
\label{chap_python_lib}

This chapter shows how you can use the \nutils{}'s functionalities in Python
programs. While the \nutils{} are written in C, the \texttt{ctypes}
module\footnote{Available in Python 2.5 and up} makes it easy to access them
from Python, and the distribution contains a module, \texttt{newick\_utils.py},
that provides an object-oriented interface to the underlying C code.

Let's say we want to add a utility that prints simple statistics about trees,
like the number of nodes, the depth, whether it is a cladogram or a phylogram,
etc. We will call it \texttt{nw\_info.py}, and we'll pass it a \nw{}
file on standard input, so the usage will be something like:

\begin{verbatim}
$ nw_info.py < data/catarrhini
\end{verbatim}

\noindent{}The overall structure of this program is simple: iteratively read
all the input trees, and do something with each of them:

\begin{lstlisting}
from newick_utils import *

for tree in Tree.parse_newick_input():
    pass # process tree here!
\end{lstlisting}

\noindent{}Line 1 imports definitions from the \texttt{newick\_utils.py}
module. Line 2 is the main loop: the \texttt{Tree.parse\_newick\_input}
reads standard input and yields an instance of class \texttt{Tree} for each
Newick string. We can now work with it, using methods of class \texttt{Tree} or adding our own:

\lstinputlisting{../src/nw_info.py}

\noindent{}When we run the program, we get:

\verbatiminput{python_1_txt.cmd}
\verbatiminput{python_1_txt.out}

As you can see, most of the work is done by methods called on the \texttt{tree}
object, such as \texttt{get\_leaf\_count} which (surprise!) returns the number
of leaves of a tree. But since there is no method for coounting polytomies, we
added our own function, \texttt{count\_polytomies}, which takes a \texttt{Tree}
object as argument.

Detailed information about all classes and methods is found in file
\texttt{newick\_utils.py}.
