%&context

\section[nodes_vs_depth]{Number of nodes vs. Tree Depth}


A simple measure of a tree's shape can be obtained by computing the number of
nodes as a function of depth. Consider the following trees:

\starttabulate[|c|c|c|]
\NC star \NC balanced \NC short\_leaves \NC\NR
% \HL
% \NC \externalfigure[nodes_vs_depth_1_svg] \NC \externalfigure[nodes_vs_depth_2_svg] \NC \externalfigure[nodes_vs_depth_3_svg] \NC\NR
\stoptabulate

they have the same depth and the same number of
leaves.  But their shapes are very different, and they tell different
biological stories. If we assume that they are clock-like (i.e., that the
mutation rate is constant over the whole tree), \id{star} shows an early
radiation, \id{short\_leaves} shows two stable lineages ending in
recent branching, while \id{balanced} shows branching spread over time.

The nodes-vs-depth graphs for these trees are as follows: \\
\externalfigure[star][width=12cm] \\
\externalfigure[balanced][width=12cm] \\
\externalfigure[short_leaves][width=12cm] \\

The graphs show the (normalized) area under the curve: it is close to 1 for star-like trees, close to 0 for trees with very short leaves, and intermediary for more balanced trees.

The images were made with the \filename{nodes\_vs\_clades.sh} script (in
directory \filename{src}), in the following way:
\starttyping
$ nodes_vs_clades.sh star 40
\stoptyping
where 40 is just the sampling density (how many points to take on the $x$
axis). The script uses \distance{} to get the tree's depth, \ed{} to sample the
number of nodes at a given depth, and \nwindent{} to count the leaves, plus the
usual \progname{awk} and friends. The plot is done with gnuplot.
