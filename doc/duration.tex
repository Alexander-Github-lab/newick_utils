\section{Converting Node Ages to Durations}
\label{sct_duration}

Sometimes you have information about the \textit{age} of a node rather than the
length of its branch. Consider the following phylogeny of major chordate groups:

\begin{center}
 \includegraphics{duration_1_svg.pdf}
\end{center}

Suppose we have the following information about the age of certain events:

\smallskip
\begin{tabular}{lr}
\textbf{event} & \textbf{age} (million years ago)\\
\hline
split of vertebrates into gnathostomes and conodonts & 530 \\
extinction of conodonts & 200 \\
split of chordates into vertebrates and urochordates & 540 \\
\end{tabular}
\smallskip

\noindent{}We can use the branch length field of Newick to specify ages, like
this:

\begin{samepage}
\verbatiminput{duration_2_nw.cmd}
\verbatiminput{duration_2_nw.out}
\end{samepage}

\noindent{}Note that the \texttt{Gnathostomata} leaf has no age: this means it
did not go extinct (you and I are gnathostomes\footnote{Well, at least \textit{I} am one}); the same goes for urochordates.

Now, if we were to display this tree without further ado, it would be nonsense.
We have to convert the ages into durations, and this is the function of
\duration{}:

\verbatiminput{duration_3_svg.cmd}
\begin{center}
 \includegraphics{duration_3_svg.pdf}
\end{center}

\noindent{}We can improve the scale bar by supplying option \texttt{-t} to
\display{}: this aligns the origin of the scale bar with the leaves and counts
backwards. To top it off, we'll specify the units as million years ago:

\verbatiminput{duration_4_svg.cmd}
\begin{center}
 \includegraphics{duration_4_svg.pdf}
\end{center}


