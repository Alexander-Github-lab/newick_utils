\section{Computing Bootstrap Support}

\support{} computes bootstrap support values from a target tree and a file of
replicate trees. Say the target tree is in file \filename{HRV.nw} and the
replicates (20 of them) are in \filename{HRV\_20reps.nw}. You can attribute
support values to the target tree like this:

\svgCmdOutput{support_1}

In this case I have colored the support values red. Option \code{-p} uses
percentages instead of absolute counts.

\subsection{Notes}

There are many tree-building programs that compute bootstrap support. For
example, PhyML can do it, but for large tasks I typically have to distribute
the replicates over several jobs (say, 100 jobs of 10 replicates each). I then
collect all replicates files, concatenate them, and use \support{} to attribute
the values to the target tree.

\support{} assumes rooted trees (it may as well, since \nw{} is implicitly
rooted), and the target tree and replicates should be rooted the same way. Use
\reroot{} to ensure this.
