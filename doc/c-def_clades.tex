%&context

\chapter[sct_def_clades]{Defining Clades by their Descendants}

When you need to specify a clade using the \nutils{}, you either give the label
of the clade's root, or the labels of (some of) its descendants. Since inner
nodes rarely have labels (or worse, have unusable labels like bootstrap
support values), you will often need to specify clades by their descendants.
Consider the following tree:

\startalignment[center]
\externalfigure[clade_2_svg] 
\stopalignment

Suppose we want to specify the Hominoidea clade - the apes. It is
the clade that contains \id{Homo}, \id{Pan} (chimps), \id{Gorilla},
\id{Pongo} (orangutan), and \id{Hylobates} (gibbons). The clade is not
labeled in this tree, but this list of labels defines it without ambiguity. In
fact, we can define it unambiguously using just \id{Hylobates} and
\id{Homo} - or \id{Hylobates} and any other label. The point is that
{\em you never need more than two labels to unambiguously define a clade}.

You cannot choose any two nodes, however: the condition is that the last
common ancestor of the two nodes be the root of the desired clade. For
instance, if you used \id{Pongo} instead of \id{Hylobates}, you would
define the Hominidae clade, leaving out the gibbons.

\section{Why not just use node numbers?}

Some applications attribute numbers to all inner nodes and allow users to
specify clades by referring to this number. Such a scheme is not workable when
one has many input trees, however, because there is no guarantee that the same
clade (assuming it is present) will have the same number in different trees.
