%&context
\section[sct_prune]{Pruning}


Pruning is simply removing arbitrary nodes. Say you have the following tree (as
it happens, it contains a glaring error since the sister clade of mammals is the
amphibian rather than the bird):

\svgFigure{prune_1}

and say you only need a subset of the species, perhaps because you
want to compare this tree to another tree with fewer species. Specifically,
let's say you don't need to show \sciname{Tetraodon}, \sciname{Danio},
\sciname{Bombina}, and \sciname{Didelphis}. You just pass those labels to
\prune:

\svgCmdOutput{prune_2}

Note that each label is removed individually. The discarding of
\sciname{Didelphis} is the cause of the disappearance of the node labeled
Mammalia. And the embarrassing error is hidden by the removal of
\sciname{Bombina}.

You can also discard internal nodes, if they are labeled (in future versions
it will be possible to discard a clade by specifying descendants, just like
\clade). For example, you can discard the whole mammalian clade like this:

\svgCmdOutput{prune_3}

By the way, \sciname{Tetrao} and \sciname{Tetraodon} are not the same thing, the
first is a bird (grouse), the second is a pufferfish.

\subsection{Keeping selected Nodes}

By passing option \code{-v} (think of \code{grep -v}), the nodes whose
labels are passed are {\em kept}, and the other ones are discarded (except
unlabeled nodes). And I really mean this: if a node's label is not on the
command line, it goes away, even if it is an inner node - this can have
surprising results.\footnote{In future versions there will be an option for
finer control of this behaviour}.

Suppose I think that the tree is unfairly biased towards mammals (and in a
lesser way, actinopterygians), and want to keep only the following genera:
\sciname{Fugu}, \sciname{Tetrao}, \sciname{Bombina}, \sciname{Sorex}. I could, of
course, generate the list of all leaves that must go away, but it is easier to
do this:

\svgCmdOutput{prune_4}

Note that I also passed \id{Mammalia}, for the reason discussed
above: the node with this label would go away if I did not, resulting in a
different tree (try it out).
