\section{Stream editing}
\label{sct_ed}

\subsection{Background}

The programs we have seen so far are all specialized for a given task,
hopefully one of the more frequent ones. It is, of course, impossible to
foresee every way in which one may need to process a tree, and with this we hit
a limit to specialization. In some cases, a more general-purpose program may
offer a solution.

The programs described in this section fall into that category: they are less
specialized, usually less efficient, but more flexible than the ones shown up
to now. They were inpired by \unix{} utilities like \texttt{sed} and
\texttt{awk}, which perform an action on the parts of input (usually lines)
that meet some condition. Rather than lines in a file, they work with nodes in
a tree: each node is visited in turn, and if it meets a user-specified
condition, a user-specified action is performed.

As a word of warning, I should say that these programs are among the more
experimental in the \nutils{} package. This is why there are two or more
programs that do basically the same thing, although differently and with
different capabilities: \ed{} is the simplest (and first), it is written
entirely in C and it is fairly limited. \sched{} was developed to address
\ed{}'s limitations: by embedding a Scheme interpreter (\gnu{} Guile), its
flexibility is, for practical purposes, limitless. Of course, this comes at the
price of linking to an external library, which may not be available. Therefore
\ed{}, for all its limitations, will stay in the package as it does not depend
on Guile. Finally, I understand that Scheme is not the only solution as an
embedded language, and my intention is to try the same approach with Lua.

\subsection{\ed}

\ed{} iterates over the nodes in a specific order, and for each node it
evaluates a logical expression provided by the user. If the expression is true,
\ed{} performs a user-specified action.  By default, the (possibly modified)
tree is printed at the end of the run.

Let's look at an example before we jump into the details. Here is a tree of vertebrate genera, showing support values:

\verbatiminput{ed_1_svg.cmd}
\begin{center}
\includegraphics{ed_1_svg.pdf}
\end{center}

\noindent{}Let's extract all well-supported clades, using a support value of 95\% or more as the criterion for being well-supported:

\verbatiminput{ed_2_txt.cmd}
\begin{samepage}
\verbatiminput{ed_2_txt.out}
\end{samepage}

\noindent{}This instructs \ed{} to iterate over the nodes, in \no{}, and to
print the subtree (action \texttt{s}) for all nodes that match the expression
\texttt{b >= 95}, which means "intepret the node label as a (bootstrap) support
value, and evaluate to true if that value is greater than 95". As we can see,
\ed{} reports the three well-supported clades (primates, tetrapods, and
ray-finned fishes), in \no. We also remark that one of the clades (primates) is
contained in another (tetrapods). Finally, option \texttt{-n} suppresses
the printing of the whole tree at the end of the run, which isn't useful here.

The two parameters of \ed{} (besides the input file) are an \emph{address} and
an \emph{action}. Addresses specify sets of nodes, and actions are performed on
them. 

\subsubsection{Addresses}

Currently, addresses are logical expressions involving node properties, and the
action is performed on the nodes for which the expression is true. They are
composed of numbers, logical operators, and node functions.

The functions have one-letter names, to keep expressions short (after all, they are passed on the command line). There are two types, numeric and boolean.

\begin{center}
\begin{tabular}{cll}
name & type & meaning \\
\hline
\texttt{a} & numeric & number of ancestors of node	 \\
\texttt{b} & numeric & node's support value (or zero) \\
\texttt{c} & numeric & node's number of children (direct) \\
\texttt{D} & numeric & node's number of descendants (includes children) \\
\texttt{d} & numeric & node's depth (distance to root) \\
\texttt{i} & boolean & true iff node is strictly internal (i.e., not root!) \\
\texttt{l} (ell) & boolean & true iff node is a leaf \\
\texttt{r} & boolean & true iff node is the root
\end{tabular}
\end{center}

The logical and relational operators work as expected, here is the list, in
order of precedence, from tightest to loosest-binding.  Anyway, you can use
parentheses to override precedence, so don't worry.

\begin{center}
\begin{tabular}{cl}
symbol & operator \\
\hline
\texttt{!} & logical negation \\
\hline
\texttt{==} & equality \\
\texttt{!=} & inequality \\
\texttt{<} & greater than \\
\texttt{>} & lesser than \\
\texttt{>=} & greater than or equal to \\
\texttt{<=} & lesser than or equal to \\
\hline
\texttt{\&} & logical and \\
\hline
\texttt{|} & logical or
\end{tabular}
\end{center}

\noindent{}Here are a few examples:

\begin{center}
\begin{tabular}{cl}
expression & selects: \\
\hline
\texttt{l} & all leaves \\
\texttt{l \& a <= 3} & leaves with 3 ancestors or less \\
\texttt{i \& (b >= 95)} & internal nodes with support greater than 95\% \\ 
\texttt{i \& (b < 50)} & unsupported nodes (less than 50\%) \\
\texttt{!r} & all nodes except the root \\
\texttt{c > 2} & multifurcating nodes
\end{tabular}
\end{center}

\subsubsection{Actions}

The actions are also coded by a single letter, for the same reason. For now,
the following are implemented:

\begin{center}
\begin{tabular}{clc}
code & effect & modifies tree?\\
\hline
\texttt{d} & delete the node (and all descendants) & yes \\
\texttt{l} & print the node's label & no \\
\texttt{o} & splice out the node & yes \\
\texttt{s} & print the subtree rooted at the node & no \\
\end{tabular}
\end{center}

\ed{} is somewhat experimental; it is also the only program that is not
deliberately "orthogonal" to the rest, that is, it can emulate some of the
functionality of other utilities.

\subsection{Opening Poorly-supported Nodes}

When a node has low support, it may be better to splice it out from the tree,
reflecting uncertainty about the true topology. Consider the following tree, 
\texttt{HRV\_cg.nw}:

\begin{center}
\includegraphics{ed_3_svg.pdf}
\end{center}

\noindent{}The inner node labels represent bootstrap support, as a percentage
of replicates. As we can see, some nodes are much better supported than others.
For example, the \texttt{(COXB2,ECHO6)} node (top of the figure) has a support
of only 35\%, and in the lower right of the graph there are many nodes with
even poorer support. Let's "open" the nodes with less than 50\% support. This means that those nodes will be spliced out, and their children attached to their "grandparent":

\verbatiminput{ed_4_svg.cmd}
\begin{center}
\includegraphics{ed_4_svg.pdf}
\end{center}

\noindent{}Now \texttt{COXB2} and \texttt{ECHO6} are siblings of
\texttt{ECHO1}, forming a node with 90\% support. What this means is that the
original tree strongly supports that these three species form a clade, but is
much less clear about the relationships \emph{within} the clade. Opening the
nodes makes this fact clear by creating multifurcations. Likewise, the lower
right of the figure is now occupied by a highly multifurcating (8 children) but
perfectly supported (100\%) node, none of whose descendants has less than 80\%
support.

\subsection{\sched}

\sched{} works in much the same way as \ed{} does, but its address (also called
{\it selector} and action are passed together as a Scheme expression.

\noindent{}As a first example, let's again extract all well-supported clades from the tree of vertebrate genera, as we did with \ed.

\verbatiminput{ed_5_txt.cmd}
\begin{samepage}
\verbatiminput{ed_5_txt.out}
\end{samepage}

\noindent{}This instructs \ed{} to iterate over the nodes, in \no{}, and to
print the subtree (action \texttt{s}) for all nodes that match the expression
\texttt{b >= 95}, which means "intepret the node label as a (bootstrap) support
value, and evaluate to true if that value is greater than 95". As we can see,
\ed{} reports the three well-supported clades (primates, tetrapods, and
ray-finned fishes), in \no. We also remark that one of the clades (primates) is
contained in another (tetrapods). Finally, option \texttt{-n} suppresses
the printing of the whole tree at the end of the run, which isn't useful here.

The two parameters of \ed{} (besides the input file) are an \emph{address} and
an \emph{action}. Addresses specify sets of nodes, and actions are performed on
them. 

\subsubsection{Addresses}

Currently, addresses are logical expressions involving node properties, and the
action is performed on the nodes for which the expression is true. They are
composed of numbers, logical operators, and node functions.

The functions have one-letter names, to keep expressions short (after all, they are passed on the command line). There are two types, numeric and boolean.

\begin{center}
\begin{tabular}{cll}
name & type & meaning \\
\hline
\texttt{a} & numeric & number of ancestors of node	 \\
\texttt{b} & numeric & node's support value (or zero) \\
\texttt{c} & numeric & node's number of children (direct) \\
\texttt{D} & numeric & node's number of descendants (includes children) \\
\texttt{d} & numeric & node's depth (distance to root) \\
\texttt{i} & boolean & true iff node is strictly internal (i.e., not root!) \\
\texttt{l} (ell) & boolean & true iff node is a leaf \\
\texttt{r} & boolean & true iff node is the root
\end{tabular}
\end{center}

The logical and relational operators work as expected, here is the list, in
order of precedence, from tightest to loosest-binding.  Anyway, you can use
parentheses to override precedence, so don't worry.

\begin{center}
\begin{tabular}{cl}
symbol & operator \\
\hline
\texttt{!} & logical negation \\
\hline
\texttt{==} & equality \\
\texttt{!=} & inequality \\
\texttt{<} & greater than \\
\texttt{>} & lesser than \\
\texttt{>=} & greater than or equal to \\
\texttt{<=} & lesser than or equal to \\
\hline
\texttt{\&} & logical and \\
\hline
\texttt{|} & logical or
\end{tabular}
\end{center}

\noindent{}Here are a few examples:

\begin{center}
\begin{tabular}{cl}
expression & selects: \\
\hline
\texttt{l} & all leaves \\
\texttt{l \& a <= 3} & leaves with 3 ancestors or less \\
\texttt{i \& (b >= 95)} & internal nodes with support greater than 95\% \\ 
\texttt{i \& (b < 50)} & unsupported nodes (less than 50\%) \\
\texttt{!r} & all nodes except the root \\
\texttt{c > 2} & multifurcating nodes
\end{tabular}
\end{center}

\subsubsection{Actions}

The actions are also coded by a single letter, for the same reason. For now,
the following are implemented:

\begin{center}
\begin{tabular}{clc}
code & effect & modifies tree?\\
\hline
\texttt{d} & delete the node (and all descendants) & yes \\
\texttt{l} & print the node's label & no \\
\texttt{o} & splice out the node & yes \\
\texttt{s} & print the subtree rooted at the node & no \\
\end{tabular}
\end{center}

\ed{} is somewhat experimental; it is also the only program that is not
deliberately "orthogonal" to the rest, that is, it can emulate some of the
functionality of other utilities.

\subsection{Opening Poorly-supported Nodes}

When a node has low support, it may be better to splice it out from the tree,
reflecting uncertainty about the true topology. Consider the following tree, 
\texttt{HRV\_cg.nw}:

\begin{center}
\includegraphics{ed_3_svg.pdf}
\end{center}

\noindent{}The inner node labels represent bootstrap support, as a percentage
of replicates. As we can see, some nodes are much better supported than others.
For example, the \texttt{(COXB2,ECHO6)} node (top of the figure) has a support
of only 35\%, and in the lower right of the graph there are many nodes with
even poorer support. Let's "open" the nodes with less than 50\% support. This means that those nodes will be spliced out, and their children attached to their "grandparent":

\verbatiminput{ed_4_svg.cmd}
\begin{center}
\includegraphics{ed_4_svg.pdf}
\end{center}

\noindent{}Now \texttt{COXB2} and \texttt{ECHO6} are siblings of
\texttt{ECHO1}, forming a node with 90\% support. What this means is that the
original tree strongly supports that these three species form a clade, but is
much less clear about the relationships \emph{within} the clade. Opening the
nodes makes this fact clear by creating multifurcations. Likewise, the lower
right of the figure is now occupied by a highly multifurcating (8 children) but
perfectly supported (100\%) node, none of whose descendants has less than 80\%
support.
