\section{Stream-editing}
\label{sct_ed}

\ed{} is one of the more experimental programs in the \nutils{} package. It was
inpired by \unix{} utilities like \texttt{sed} and \texttt{awk}, which perform
an action on the parts of input (usually lines) that meet some condition: 
\ed{} iterates over tree nodes, evaluates a user-provided expression involving
the node, and perfoms the specified action on those nodes that match. By
default, the (possibly modified) tree is printed at the end of the run.

Let's look at an example before we jump into the details. Here is a tree of vertebrate genera, showing support values:

\verbatiminput{ed_1_svg.cmd}
\begin{center}
\includegraphics{ed_1_svg.pdf}
\end{center}

\noindent{}Let's extract all well-supported clades, using a support value of 95\% or more as the criterion for being well-supported:

\verbatiminput{ed_2_txt.cmd}
\verbatiminput{ed_2_txt.out}

\noindent{}This instructs \ed{} to iterate over the nodes, in \no{}, and to
print the subtree (action \texttt{s}) for all nodes that match the expression
\texttt{b >= 95}, which means "intepret the node label as a (bootstrap) support
value, and evaluate to true if that value is greater than 95". As we can see,
\ed{} reports the three well-supported clades (primates, tetrapods, and
ray-finned fishes), in \no. We also remark that one of the clades
(primates) is contained in another one (tetrapods). Finally, option \texttt{-n}
suppresses the printing of the whole tree at the end of the run, which isn't
useful here.

\subsection{Opening Poorly-supported Nodes}
