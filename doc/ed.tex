\section{Stream editing}
\label{sct_ed}

\subsection{Background}

The programs we have seen so far are all specialized for a given task,
hopefully one of the more frequent ones. It is, of course, impossible to
foresee every way in which one may need to process a tree, and with this we hit
a limit to specialization. In some cases, a more general-purpose program may
offer a solution.

As an analogy, consider the task of finding lines in a text file that match a
given pattern. This can be done, for example, in the following ways, from the
general to the specific:
\begin{itemize}
	\item a Perl program
	\item a {\tt sed} one-liner
	\item a {\tt grep} command
\end{itemize}
Perl is a general-purpose language, it just happens to be rather good at
processing text.\footnote{Ok, Perl was \emph{initially} designed for processing
text -- it's the Practical Extraction and Report Language, after all -- but it
has long grown out of this initial specialization.} Sed is specialized for
editing text streams, and {\tt grep} is designed for precisely the line-finding
task in question.\footnote{The name "grep" comes from the {\tt sed} expression
{\tt g/re/p}, where "re" stands for "regular expression".} We should expect
{\tt grep} to be the most efficient, but we should not expect it to be able to
perform any significantly different task.  By contrast, Perl may be (I haven't
checked!) less efficient than {\tt grep}, but it can handle pretty much any
task. Sed will lie in between. The programs we have seen so far are {\tt
grep}-like: they are specialized for one task (and hopefully, they are
efficient).

The programs described in this section are more {\tt sed}-like: they are less
specialized, usually less efficient, but more flexible than the ones shown up to
now. They were in fact inspired by \texttt{sed} and \texttt{awk}, which perform
an action on the parts of input (usually lines) that meet some condition. Rather
than lines in a file, the programs presented here work with nodes in a tree:
each node is visited in turn, and if it meets a user-specified condition, a
user-specified action is performed. In other words, they are node-oriented
stream editors for trees.

As a word of warning, I should say that these programs are among the more
experimental in the \nutils{} package. This is why there are three programs that
do basically the same thing, although differently and with different
capabilities: \ed{} is the simplest (and first), it is written entirely in C and
it is fairly limited. \sched{} was developed to address \ed{}'s limitations: by
embedding a Scheme (\url{http://www.r6rs.org}) interpreter (\gnu{} Guile,
\url{http://www.gnu.org/software/guile/}), its flexibility is, for practical
purposes, limitless. Of course, this comes at the price of linking to an
external library, which may not be available. Therefore \ed{}, for all its
limitations, will stay in the package as it has no external dependency. Finally,
I understand that Scheme is not the only solution as an embedded language, and
that many people (myself included) find learning it a bit of a challenge.
Therefore, I tried the same approach with Lua\footnote{In case you're curious,
the reason I tried Scheme before Lua is simply that I heard about them in that
order.} (\url{http://www.lua.org}), which is designed as an embeddable language,
is even smaller than Guile, and by most accounts easier to learn.\footnote{And,
in my experience, easier to embed in a C program, but your experience may
differ. In particular, I could provide all of \luaed's functionality without
writing a single line of Lua code, whereas \sched{} relies on a few dozen lines
of embedded Scheme code that have to be parsed and interpreted on each run. But
that may very possibly just reflect my poor Scheme/Guile skills.} The result,
\luaed, is probably the best so far: as powerful as \sched, while smaller,
faster and easier to use. For this reason, I will probably not develop \sched{}
much more, but I won't drop it altogether either, not soon at any rate. 

\subsection{The General Idea}

\ed, \sched, and \luaed{} work in the same way: they iterate over the nodes in a
specific order (\nw{} order by default), and for each node they evaluate a
logical expression provided by the user. If the expression is true, they perform
a user-specified action.  By default, the (possibly modified) tree is printed at
the end of the run.  

Where the programs differ is the way the expression and action are expressed; as
well as the expressive power of the languages used; some of them can also
perform actions at the start or end of the run, or of each input tree. These are
summarized in the table below.

\begin{center}
	\begin{minipage}{0.8\textwidth}
		% see http://ostraya.livejournal.com/250833.html for the "multiple refs to
		% same footnote in a minipage" trick
		\renewcommand{\thefootnote}{\thempfootnote}
		\begin{tabular}{llll}
		& \ed & \sched & \luaed \\
		\hline
		language & own & Scheme & Lua \\
		programming constructs & no & Scheme's & Lua's \\
		functions & fixed & arbitrary\footnote{\ie, user can define their own} &
		arbitrary\footnotemark[\value{mpfootnote}] \\
		depends on & nothing & \gnu{} Guile & Lua library \\
		pre- \& post-tree code & no & yes & yes \\
		pre- \& post-run code & no & yes & yes
		\end{tabular}
	\end{minipage}
\end{center}

\subsection{\luaed}

Although \luaed{} is the most recent of the three, we'll cover it first because
if this one does what you need it's unlikely you'll need the others.  Let's look
at an example before we jump into the details. Here is a tree of vertebrate
genera, showing support values:

\verbatiminput{ed_1_svg.cmd}
\begin{center}
\includegraphics{ed_1_svg.pdf}
\end{center}

\noindent{}Let's extract all well-supported clades, using a support value of
95\% or more as the criterion for being well-supported. In our jargon, the
\emph{condition} would be that a node i) have a support value in the first place
(some nodes don't, \eg{} the root and the \lca{} of (Fugu,Tetraodon)), and ii)
that this value be no less than 95. The \emph{action} would simply be to print
out the tree rooted at the current node.

\verbatiminput{ed_10_txt.cmd}
\begin{samepage}
\verbatiminput{ed_10_txt.out}
\end{samepage}

\noindent{}As always, the first argument to the program is a tree file,
\texttt{vrt2cg.nw} in this example.

The second argument, \verb+b ~= nil and b >= 95+, is the condition.  In this
case, it is just the conjunction (\texttt{and}) of two expressions
\verb+b ~= nil+ and \texttt{b >= 95}.  The former checks that the node's support
value (variable \texttt{b}) is defined (\ie, not \texttt{nil}); the latter
checks that the support value is no less than 95. Note that the checks occur in
that order, and that if \texttt{b} isn't defined, the second check isn't even
performed, as it is meaningless.

The third argument, \texttt{s()}, is the action: it specifies what to do when a
node meets the condition -- in this case, call function \texttt{s},
which just prints the tree rooted at the current node.

\subsubsection{Conditions}

Conditions are Boolean expressions usually involving node properties which are
available as predefined variables. As the program "visits" each node in turn,
the variables are set to the current node's properties.  These predefined
variables have short names, to keep expressions concise. They are shown in table
\ref{tbl:luaedPredefVar}.

\begin{table}
	\centering
	\begin{tabular}{cll}
		name & (Lua) type & meaning (refers to the current node)\\
		\hline
		\texttt{a} & integer & number of ancestors  \\
		\texttt{b} & number & support value (or \texttt{nil}) \\
		\texttt{c} & integer & number of children (direct descendants) \\
		\texttt{D} & integer & total number of descendants (includes children) \\
		\texttt{d} & number & depth (distance to root) \\
		\texttt{i} & Boolean & true iff node is strictly internal (i.e., not root!) \\
		\texttt{lbl} & string & label \\
		\texttt{l} (ell) & Boolean & true iff node is a leaf \\
		\texttt{L} & number & parent edge length \\
		\texttt{N} & node & the current node itself \\
		\texttt{r} & Boolean & true iff node is the root
	\end{tabular}
	\caption{%
		\label{tbl:luaedPredefVar}
		Predefined variables in \luaed. Variables \texttt{b} and \texttt{lbl} are
		both derived from the label, but \texttt{b} is interpreted as a number, and
		is undefined if the conversion to a number fails, or if the node is a leaf.
		Edge length and depth (\texttt{L} and \texttt{d}) are undefined (not zero!)
		if not specified in the Newick tree, as in cladograms.
	}
\end{table}

The condition being just a Boolean expression written in Lua, all the logical
operators of the Lua language can be used (indeed, any valid Lua snippet can be
used, provided it evaluates to a Boolean), and you can use parentheses to
override operator precedence or for clarity.

\noindent{}Here are some examples of \luaed{} conditions:

\begin{center}
\begin{tabular}{cl}
expression & selects: \\
\hline
\texttt{l} (lowercase ell) & all leaves \\
\texttt{l and a <= 3} & leaves with 3 ancestors or less \\
\texttt{i and (b \~{ }= nil) and (b >= 95)} & internal nodes with support $\geq$ 95\% \\ 
\texttt{i and (b \~{ }= nil) and (b < 50)} & unsupported nodes (less than 50\%) \\
\texttt{not r} & all nodes except the root \\
\texttt{c > 2} & multifurcating nodes
\end{tabular}
\end{center}

Notes:
\begin{itemize}
	\item If it is certain that all nodes do have support, checks such as \verb+b ~= nil+ can be omitted. 
	\item if an action must be performed on every node, just pass \texttt{true} as
	the condition.
\end{itemize}

\subsubsection{Actions}

Actions are arbitrary Lua expressions. These will typically involve printing out
data or altering node properties or even tree structure. \luaed{} predefines a
few functions for such purposes (table \ref{tbl:luaedPredefFunc}), and you can
of course write your own (unless the function is very short, this is easier if
you pass the Lua code in a file, see \ref{sct:lua_file_input}).

\begin{table}
	\centering
	\begin{tabular}{clc}
		code & effect & modifies tree?\\
		\hline
		\texttt{o}	& splice out the node 									& yes \\
		\texttt{s}	& print the subtree rooted at the node 	& no \\
		\texttt{u} 	& delete ("unlink") the node (and all descendants) & yes
	\end{tabular}
	\caption{%
		\label{tbl:luaedPredefFunc}
		Predefined actions in \luaed. The names are one letter long for convenience
		when passing the action on the command line. When called without an
		argument, these functions operate on the current node (\ie, \texttt{s()} is
		the same as \texttt{s(N)} (where \texttt{N} means the current node -- see
		table \ref{tbl:luaedPredefVar}).
	}
\end{table}

\sched{} defines a "node" type, and the current node is always accessible as
variable \texttt{N} (other nodes can be obtained through node properties, see
below).  Node properties can be accessed as fields in a Lua table. Table \ref{tbl:luaedNodeProp} lists the available node
fields.

\begin{table}
	\center
	\begin{tabular}{llcl}
		name & type & mode & meaning \\
		\hline
		\texttt{len}, \texttt{L} 	& number 	& rw & parent edge's length \\
		\texttt{lbl}							& string 	& rw & label \\
		\texttt{b}								& number	& ro & support value \\
		\texttt{par}							& node		& ro & parent \\
		\texttt{first\_child}, \texttt{fc}  & node & ro & first child \\
		\texttt{last\_child}, \texttt{lc} 	& node & ro & last child \\
		\texttt{children\_count}, \texttt{c} 	& integer & ro & number of children \\
		\texttt{kids}							& table		& ro & list of children nodes
	\end{tabular}
	\caption{%
		\label{tbl:luaedNodeProp}
		Node properties accessible from Lua. rw: read-write, ro: read only. Some
		fields have both a short and a long name, the former is intended for use on
		the command line (where space is at a premium), the latter is for use in
		scripts (but both can be used anywhere). Note that when referring to the
		\emph{current} node, the predefined variables (see table
		\ref{tbl:luaedPredefVar}) are even more concise, \eg{} \texttt{N.len} or
		\texttt{N.L} can be written just \texttt{L}, but they are read-only.
	}
\end{table}

\noindent{}So for example the parent of the current node is expressed by
\texttt{N.par}; doubling its length could be \texttt{N.par.len = N.par.len * 2}.

\subsubsection{Examples}

\paragraph{Formatting Lengths} 

\noindent{}Some phylogeny programs return \nw{} trees with an unrealistic number
of decimal places. For example, the \texttt{HRV.nw} tree has six:

\verbatiminput{ed_6_txt.cmd}
\begin{samepage}
\verbatiminput{ed_6_txt.out}
\end{samepage}

\noindent{}Here I use \nwindent{} to show each node on a line for clarity, and
show only the last ten.\footnote{the \emph{first} ten lines contain only
opening parentheses.} To format\footnote{\sched{} automatically loads the {\tt
format} module so that the full-fledged {\tt format} function is available.}
the lengths to two decimal places, do the following:

\verbatiminput{ed_11_txt.cmd}
\begin{samepage}
\verbatiminput{ed_11_txt.out}
\end{samepage}

\begin{samepage}
\paragraph{Multiplying lengths by a constant}

It may be necessary to have two trees which only differ by a constant multiple
of the branch lengths. This can be used, for example, to test competing
hypotheses about evolution rates.
\end{samepage}
Here is our good friend the Catarrhinine tree again:

\verbatiminput{ed_8_svg.cmd}
\begin{center}
\includegraphics{ed_8_svg.pdf}
\end{center}

To multiply all its branch lengths by (say) 3.5, do the following:

\verbatiminput{ed_12_svg.cmd}
\begin{center}
\includegraphics{ed_12_svg.pdf}
\end{center}

\begin{samepage}
\paragraph{Implementing other \nutils}

\noindent{}\luaed{} can emulate other programs in the package, when these
iterate on every node and perform some action. There is no real reason to use
\luaed{} rather than the original, since \luaed{} will be slower (after all, it
has to start the Lua interpreter, parse the Lua expressions, etc.). But it
can serve as illustration.
\end{samepage}

\begin{center}
\begin{tabular}{ll}
program & \luaed{} equivalent \\
\hline
\labels{} 						& {\tt 'lbl \~{}= ""' 'print(lbl)'}  \\
\labels{} {\tt -I} 		& {\tt 'l and lbl \~{}= ""' 'print(lbl)'} \\
\topology{}						& {\tt true 'N.len = ""'} \\
\topology{} {\tt -I} 	& {\tt true 'N.len = ""; if not l then N.lbl = "" end'}
\end{tabular}
\end{center}

The {\tt lbl \~{}= ""} condition in the \labels{} replacements is checked
because the original \labels{} does not print empty labels. In the \topology{}
replacement, the check for node type (\texttt{l}) is done in the action rather
than the condition, because there is some code that is performed for every node
and some additional code only for non-leaves. 

\subsubsection{Lua script in a file}
\label{sct:lua_file_input}

Sometimes the command line is too short to comfortably write the condition and
action. In this case, one can put the Lua code in a file, which is passed to
\luaed{} via option \texttt{-f}. The file can contain any Lua code, but some
function names are special: they function as "hooks", that is, they are called
on predefined occasions. Table \ref{tbl:luaedHooks} shows the hook names and
when they are called.

\begin{table}
	\centering
	\begin{tabular}{ll}
		hook name & called\ldots \\
		\hline
		\texttt{start\_run} 	& before processing any tree \\
		\texttt{start\_tree}	& for each tree, before processing \\
		\texttt{node}					& for each node \\
		\texttt{stop\_tree}		& for each tree, after processing \\
		\texttt{stop\_run}			& after processing all trees
	\end{tabular}
	\caption{%
		\label{tbl:luaedHooks}
		Hooks defined by \luaed. If a function named \texttt{start\_tree} is
		defined, it will be called once per tree, before the tree is processed; etc.
		If a hook is not defined, no action is performed on the corresponding
		occasion. Strictly speaking, \texttt{start\_run} is not really necessary, as
		the file is evaluated before the run anyway, but it seems cleaner to provide
		a start-of-the-run hook as well.
	}
\end{table}

\paragraph{A tree counter}

As you know by now, the \nutils{} are able to process files that contain any
number of trees. But just how many trees are there in a file? If you're certain
that there is exactly one tree per line, you just use \texttt{wc -l}. But the
\nw{} format allows trees to span more than one line, or conversely there may be
more than one tree per line; moreover there may be blank lines. All these
conspire to yield wrong tree counts. To solve this, we write a tree counter in
Lua, and pass it to \luaed. Here is the counter:

\begin{samepage}
\verbatiminput{ed_13_txt.cmd}
\verbatiminput{ed_13_txt.out}
\end{samepage}

As you can see, I've defined three of the five possible hooks. Before any tree
is processed, \texttt{start\_run} is called, which defines variable
\texttt{count} and initializes it to zero. After each tree is processed
(actually, no processing is done, since the \texttt{node} hook is not defined),
function \texttt{stop\_tree} is called, which increments the counter. And after
the last tree has been processed, the \texttt{stop\_run} hook is called, which
just prints out the count.

Here it is in action. First, the simple case of one tree per line:

\verbatiminput{ed_14_txt.cmd}
\verbatiminput{ed_14_txt.out}

\verbatiminput{ed_15_txt.cmd}
\verbatiminput{ed_15_txt.out}

\noindent{}Right. Now how about this one: these are the same trees as in
\texttt{forest}, but all on a single line:

\verbatiminput{ed_16_txt.cmd}
\verbatiminput{ed_16_txt.out}

\verbatiminput{ed_17_txt.cmd}
\verbatiminput{ed_17_txt.out}

\noindent{}\luaed{} is not fooled! And this is the opposite case -- an indented tree, which has one \emph{node} per line:

\verbatiminput{ed_18_txt.cmd}
\verbatiminput{ed_18_txt.out}

\verbatiminput{ed_19_txt.cmd}
\verbatiminput{ed_19_txt.out}

There's no confusing our tree counter, it seems. Note that in future versions I
might well make this unnecessary by supplying a predefined variable which counts
the input trees, akin to Awk's \texttt{NR}.

\paragraph{Numbering inner nodes}

I was once handed a tree with the task of numbering the inner nodes, starting
close to the leaves and ending at the root.\footnote{Thanks to Beno\^{i}t
Defaucheux for this example.} Here is a tree with unlabeled inner nodes (I hide
the branch lengths lest they obscure the inner node labels, which will also be
numeric):

\verbatiminput{ed_22_svg.cmd}
\begin{center}
\includegraphics{ed_22_svg.pdf}
\end{center}

A solution is the following \luaed{} script:

\verbatiminput{ed_20_txt.cmd}
\verbatiminput{ed_20_txt.out}

\verbatiminput{ed_21_svg.cmd}
\begin{center}
\includegraphics{ed_21_svg.pdf}
\end{center} 

%% TODO: slim down the nw_ed and nw_sched sections, as most of the context is
%% introduced with nw_luaed.

\subsection{\ed}

\ed{} iterates over the nodes in a specific order, and for each node it
evaluates a logical expression provided by the user. If the expression is true,
\ed{} performs a user-specified action.  By default, the (possibly modified)
tree is printed at the end of the run.

Let's look at an example before we jump into the details. Here is a tree of vertebrate genera, showing support values:

\noindent{}Let's extract all well-supported clades, using a support value of 95\% or more as the criterion for being well-supported:

\verbatiminput{ed_2_txt.cmd}
\begin{samepage}
\verbatiminput{ed_2_txt.out}
\end{samepage}

\noindent{}This instructs \ed{} to iterate over the nodes, in \no{}, and to
print the subtree (action \texttt{s}) for all nodes that match the expression
\texttt{b >= 95}, which means "interpret the node label as a (bootstrap) support
value, and evaluate to true if that value is greater than 95". As we can see,
\ed{} reports the three well-supported clades (primates, tetrapods, and
ray-finned fishes), in \no. We also remark that one of the clades (primates) is
contained in another (tetrapods). Finally, option \texttt{-n} suppresses
the printing of the whole tree at the end of the run, which isn't useful here.

The two parameters of \ed{} (besides the input file) are an \emph{address} and
an \emph{action}. Addresses specify sets of nodes, and actions are performed on
them. 

\subsubsection{Addresses}

Currently, addresses are logical expressions involving node properties, and the
action is performed on the nodes for which the expression is true. They are
composed of numbers, logical operators, and node functions.

The functions have one-letter names, to keep expressions short (after all, they are passed on the command line). There are two types, numeric and Boolean.

\begin{center}
\begin{tabular}{cll}
name & type & meaning \\
\hline
\texttt{a} & numeric & number of ancestors of node	 \\
\texttt{b} & numeric & node's support value (or zero) \\
\texttt{c} & numeric & node's number of children (direct) \\
\texttt{D} & numeric & node's number of descendants (includes children) \\
\texttt{d} & numeric & node's depth (distance to root) \\
\texttt{i} & Boolean & true iff node is strictly internal (i.e., not root!) \\
\texttt{l} (ell) & Boolean & true iff node is a leaf \\
\texttt{r} & Boolean & true iff node is the root
\end{tabular}
\end{center}

The logical and relational operators work as expected, here is the list, in
order of precedence, from tightest to loosest-binding.  Anyway, you can use
parentheses to override precedence, so don't worry.

\begin{center}
\begin{tabular}{cl}
symbol & operator \\
\hline
\texttt{!} & logical negation \\
\hline
\texttt{==} & equality \\
\texttt{!=} & inequality \\
\texttt{<} & greater than \\
\texttt{>} & lesser than \\
\texttt{>=} & greater than or equal to \\
\texttt{<=} & lesser than or equal to \\
\hline
\texttt{\&} & logical and \\
\hline
\texttt{|} & logical or
\end{tabular}
\end{center}

\noindent{}Here are a few examples:

\begin{center}
\begin{tabular}{cl}
expression & selects: \\
\hline
\texttt{l} & all leaves \\
\texttt{l \& a <= 3} & leaves with 3 ancestors or less \\
\texttt{i \& (b >= 95)} & internal nodes with support greater than 95\% \\ 
\texttt{i \& (b < 50)} & unsupported nodes (less than 50\%) \\
\texttt{!r} & all nodes except the root \\
\texttt{c > 2} & multifurcating nodes
\end{tabular}
\end{center}

\subsubsection{Actions}

The actions are also coded by a single letter, for the same reason. For now,
the following are implemented:

\begin{center}
\begin{tabular}{clc}
code & effect & modifies tree?\\
\hline
\texttt{d} & delete the node (and all descendants) & yes \\
\texttt{l} & print the node's label & no \\
\texttt{o} & splice out the node & yes \\
\texttt{s} & print the subtree rooted at the node & no \\
\end{tabular}
\end{center}

\ed{} is somewhat experimental; it is also the only program that is not
deliberately "orthogonal" to the rest, that is, it can emulate some of the
functionality of other utilities.

\subsection{Opening Poorly-supported Nodes}
\label{sct:ed_ed}

When a node has low support, it may be better to splice it out from the tree,
reflecting uncertainty about the true topology. Consider the following tree, 
\texttt{HRV\_cg.nw}:

\begin{center}
\includegraphics{ed_3_svg.pdf}
\end{center}

\noindent{}The inner node labels represent bootstrap support, as a percentage of
replicates. As we can see, some nodes are much better supported than others.
For example, the \texttt{(COXB2,ECHO6)} node (top of the figure) has a support
of only 35\%, and in the lower right of the graph there are many nodes with even
poorer support. Let's "open" the nodes with less than 50\% support. This means
that those nodes will be spliced out, and their children attached to their
"grandparent":

\verbatiminput{ed_4_svg.cmd}
\begin{center}
\includegraphics{ed_4_svg.pdf}
\end{center}

\noindent{}Now \texttt{COXB2} and \texttt{ECHO6} are siblings of
\texttt{ECHO1}, forming a node with 90\% support. What this means is that the
original tree strongly supports that these three species form a clade, but is
much less clear about the relationships \emph{within} the clade. Opening the
nodes makes this fact clear by creating multifurcations. Likewise, the lower
right of the figure is now occupied by a highly multifurcating (8 children) but
perfectly supported (100\%) node, none of whose descendants has less than 80\%
support.

\subsection{\sched}

As mentioned above, \sched{} works like \ed{}, but embeds a Scheme interpreter
so that the address (also called the {\em selector}) and action are passed as a
Scheme expression. The Scheme language (see link above) has a simple syntax,
but it can be slightly surprising at first. To understand the following
examples, you just need to know that operators \emph{precede} their arguments,
as do function names, so that the sum of 2 and 2 is written \texttt{(+ 2 2)},
the sine of $x$ is \texttt{(sin x)}, \texttt{(< 3 2)} is false, etc.

\noindent{}As a first example, let's again extract all
well-supported clades from the tree of vertebrate genera, as we did with \ed.

\verbatiminput{ed_5_txt.cmd}
\begin{samepage}
\verbatiminput{ed_5_txt.out}
\end{samepage}

The expression \texttt{((\& (def? 'b) (>= 95 b)) (s))} parses as follows:
\begin{itemize}
	\item the first element (or \texttt{car}, in Scheme parlance),
	\texttt{(\& (def? 'b) (>= 95 b))}, is the selector.  It is a Boolean
	expression, namely a conjunction (\texttt{\&})\footnote{\texttt{\&} is
	a short name for the Scheme form \texttt{and}, which is defined by
	\sched{} to allow for shorter expressions on the command line.} of the
	expressions \texttt{(def? 'b)} and \texttt{(>= 95 b)}.  The former
	checks that variable \texttt{b} (bootstrap support) is
	defined\footnote{In \ed, \texttt{b} was zero if the support was not
	defined. \sched{} distinguishes between undefined and zero, which is
	why one has to check that \texttt{b} is defined before using it.
	\texttt{def?} is just a shorter name for \texttt{defined?}.}, and the
	latter is true iff \texttt{b} is not smaller than 95.
	\item the second element (\texttt{cadr} in Scheme jargon),
	\texttt{(s)}, is the action -- in this case, a call to function
	\texttt{s}, which has the same meaning as action \texttt{s} in \ed,
	namely to print out the subclade rooted at the current node.
\end{itemize}

\subsubsection{Selectors}

Like \ed{} addresses, \sched{} selectors are Boolean expressions normally
involving node properties which are available as predefined variables. As the
program "visits" each node in turn, the variables are set to reflect the current
node's properties. As in \ed{}, the variables have short names, to keep
expressions concise. The predefined variables are shown in the table below.

\begin{center}
\begin{tabular}{cll}
name & type & meaning \\
\hline
\texttt{a} & integer & number of ancestors  \\
\texttt{b} & rational & support value \\
\texttt{c} & integer & number of children (direct descendants) \\
\texttt{D} & integer & total number of descendants (includes children) \\
\texttt{d} & numeric & depth (distance to root) \\
\texttt{i} & Boolean & true iff node is strictly internal (i.e., not root!) \\
\texttt{lbl} & string & label \\
\texttt{l} (ell) & Boolean & true iff node is a leaf \\
\texttt{L} & rational & parent edge length \\
\texttt{r} & Boolean & true iff node is the root
\end{tabular}
\end{center}

\noindent{}Variables \texttt{b} and \texttt{lbl} are both derived from the
label, but \texttt{b} is interpreted as a number, and is undefined if the
conversion to a number fails, or if the node is a leaf. Edge length and depth
(\texttt{L} and \texttt{d}) are undefined (not zero!) if not specified in the
Newick tree, as in cladograms.

Whereas \ed{} defines logical and relational operators, \sched{} just uses those
of the Scheme language. It just defines a few shorter names to help keep command
lines compact:

\begin{center}
\begin{tabular}{lcl}
Scheme & \sched{} short form & meaning \\
\hline
\texttt{not} 	& \texttt{!} 				& logical negation \\
\texttt{and} 	& \texttt{\&} 			& logical and \\
\texttt{or}  	& \texttt{|} 				& logical or \\
\texttt{defined?}	& \texttt{def?}	& checks if arg is defined
\end{tabular}
\end{center}

\noindent{}Here are a the same examples as above, but for \sched:

\begin{center}
\begin{tabular}{cl}
expression & selects: \\
\hline
\texttt{l} (lowercase ell) & all leaves \\
\texttt{(\& l (<= a 3))} & leaves with 3 ancestors or less \\
\texttt{(\& i (def? 'b) (>= b 95))} & internal nodes with support greater than 95\% \\ 
\texttt{(\& i (def? 'b) (< b 50)} & unsupported nodes (less than 50\%) \\
\texttt{(! r)} & all nodes except the root \\
\texttt{(> c 2)} & multifurcating nodes
\end{tabular}
\end{center}

\noindent{}When it is clear that all inner nodes will have a defined support
value, one can leave out the \texttt{(def? 'b)} clause.

\subsubsection{Actions}

Actions are arbitrary Scheme expressions, so they are much more flexible than
the fixed actions defined by \ed. \sched{} defines most of them, as well as a
few new ones, as Scheme functions\footnote{Note that you must use Scheme's function call syntax to call the function, \ie, \texttt{(function [args...])}.}:

\begin{center}
\begin{tabular}{clc}
code & effect & modifies tree?\\
\hline
\texttt{L! <len>}	& sets the node's parent-edge length to \texttt{len}	& yes \\
\texttt{lbl! <lbl>}	& sets the node's label to \texttt{lbl}	& yes \\
\texttt{o} 				& splice out the node 									& yes \\
\texttt{p <arg>} 	& print \texttt{arg}, then a newline 		& no \\
\texttt{s} 				& print the subtree rooted at the node 	& no \\
\texttt{u} 				& delete ("unlink") the node (and all descendants) & yes \\
\end{tabular}
\end{center}

\noindent{}The \texttt{l} action of \ed{}, which prints the current node's
label, can be achieved in \sched{} with the more general \texttt{p} function:
\texttt{(p lbl)}.

The \texttt{L!} function sets the current node's parent-edge length. It accepts
a string or a number. If the argument is a string, it attempts to convert it to
a number. If this fails, the edge length is undefined. The \texttt{lbl!}
function sets the current node's label. Its argument is a string.

The delete action was renamed to \texttt{u} because the \texttt{d} symbol stands
for the node's depth.

\subsubsection{Examples}

{\it Formatting Lengths} 

\noindent{}Some phylogeny programs return \nw{} trees with an unrealistic number
of decimal places. For example, the \texttt{HRV.nw} tree has six:

\verbatiminput{ed_6_txt.cmd}
\begin{samepage}
\verbatiminput{ed_6_txt.out}
\end{samepage}

\noindent{}Here I use \nwindent{} to show each node on a line for clarity, and
show only the last ten.\footnote{the \emph{first} ten lines contain only
opening parentheses.} To format\footnote{\sched{} automatically loads the {\tt
format} module so that the full-fledged {\tt format} function is available.}
the lengths to two decimal places, do the following:

\verbatiminput{ed_7_txt.cmd}
\begin{samepage}
\verbatiminput{ed_7_txt.out}
\end{samepage}

\begin{samepage}
\paragraph{Multiplying lengths by a constant}

It may be necessary to have two trees which only differ by a constant multiple
of the branch lengths. This can be used, for example, to test competing
hypotheses about evolution rates.
\end{samepage}
Here is our good friend the Catarrhinine tree again:

\verbatiminput{ed_8_svg.cmd}
\begin{center}
\includegraphics{ed_8_svg.pdf}
\end{center}

To multiply all its branch lengths by (say) 3.5, do the following:

\verbatiminput{ed_9_svg.cmd}
\begin{center}
\includegraphics{ed_9_svg.pdf}
\end{center}

\begin{samepage}
\paragraph{Implementing other \nutils}

\noindent{}\sched{} can emulate other programs in the package, when these
iterate on every node and perform some action. There is no real reason to use
\sched{} rather than the original, since \sched{} will be slower (after all, it
has to start the Scheme interpreter, parse the Scheme expressions, etc.). But it
can serve as illustration.
\end{samepage}

\begin{center}
\begin{tabular}{ll}
program & \sched{} equivalent \\
\hline
\labels{} 						& {\tt '((! (string-null? lbl)) (p lbl))'}  \\
\labels{} {\tt -I} 		& {\tt '((! (| i (string-null? lbl))) (p lbl))'} \\
\topology{}						& {\tt '(\#t (L! ""))'} \\
\topology{} {\tt -I} 	& {\tt '(\#t (if l (L! "") (begin (L! "") (lbl!  ""))))'}
\end{tabular}
\end{center}

The {\tt string-null?} predicate in the \labels{} replacements is checked
because the original \labels{} does not print empty labels. The {\tt begin} form
in the \topology{} replacement is used to perform two actions: setting the
length and setting the label. Note also that the check for node type is done in
the action rather than the selector. The {\tt if} on the selector is implicit.
Future versions will allow more than one (selector, action) pair (just as Awk
does) and allow the {\tt nw\_topology -I} equivalent to be written {\tt '((\#t
(L! "")) ((! l) (lbl! "")))} -- set length to undefined for every node, and set
label to empty for all non-leaf nodes. I also intend to put an implicit {\tt
begin} around the action.
