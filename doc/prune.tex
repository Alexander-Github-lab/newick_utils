\section{Pruning}
\label{sct_prune}

Pruning is simply removing leaves or branches. Say you have the following
tree:

\includegraphics{prune_1_svg.pdf}

and say you only need a subset of the species, perhaps because you want to
compare this tree to another tree with fewer species. Specifically, let's say
you don't need to show \textit{Tetraodon}, \textit{Danio}, \textit{Bombina},
and \textit{Didelphis}. You just pass those labels to \prune:
\verbatiminput{prune_2_svg.cmd}
\includegraphics{prune_2_svg.pdf} \\
Note that each label is removed individually. The discarding of
\textit{Didelphis} is the cause of the disappearance of the node labeled
Mammalia.

You can also discard internal nodes, if they are labeled (in future versions
it will be possible to discard a clade by specifying descendants, just like
\clade). For example, you can discard the whole mammalian clade like this:
\verbatiminput{prune_3_svg.cmd}
\includegraphics{prune_3_svg.pdf} \\
By the way, \textit{Tetrao} and \textit{Tetraodon} are not the same thing, the
first is a bird (capercaillie), the second is a pufferfish.
