\clearpage
\addcontentsline{toc}{chapter}{Introduction}
\chapter*{Introduction}

The \nutils{} are a set of \unix{} (including Mac OS X) and \unix-like (Cygwin) shell programs for working with phylogenetic trees. Their main features are:
\begin{itemize}
 \item they require no user interaction\footnote{Why this is a good thing is not the focus of this document: I shall assume that if you are reading this, you already know when a command-line interface is better than an interactive interface.}
 \item they can work on any number of trees at a time
 \item they perform well with large trees
 \item they are implemented as filters\footnote{In \unix{} jargon, a
 \textit{filter} is a program that reads input from standard input and writes
 output to standard output.}
 \item they read and write text
\end{itemize}
They are not tools for \emph{making} phylogenies. Rather, they are for
processing existing ones, by which I mean manipulating the tree or extracting
information from it: rerooting, simplifying, extracting subtrees, printing
branch lengths and distances, etc - see table \ref{tbl_prog_list}; a glance
at the table of contents should also give you an idea.

Each of the programs performs one task (with some variants). For example, here
is how you would reroot a series of phylograms contained in file
\texttt{mytrees.nw} using node \texttt{Dmelano} as the outgroup:

\begin{verbatim}
$ nw_reroot mytrees.nw Dmelano
\end{verbatim} 
Now, you might want to make cladograms from the rerooted trees. Program
\topology{} does the job, and since the utilities are filters, you can do it
all in a single command:
\begin{verbatim}
$ nw_reroot mytrees.nw Dmelano | nw_topology -
\end{verbatim}
As you can see, it is straightforward to pipe \nutils{} together, and of course they can be mixed freely with any other shell program (see e.g. \ref{sct_counting_leaves}).

\subsection*{About This Document}

This tutorial is organized as follows: chapter \ref{chap_general} discusses
common features of the \nutils, chapter \ref{chap_simple} shows examples of
simple tasks, and chapter \ref{chap_adv} has examples of more advanced tasks. 

It is not necessary to read this material in order: you can pretty much jump to
any section in chapters \ref{chap_simple} and \ref{chap_adv}, they do not
require reading previous sections. I would suggest reading chapter
\ref{chap_general}, then section \ref{sct_display} because it explains how all
the tree graphics were produced.

The files for all the examples in this tutorial can be found in
subdirectory \texttt{data}.

All the program outputs are generated on-the-fly just before the document is run
through \LaTeX{}, so they represent the actual output of the latest version of
the utilities.

\begin{table}[t]
\begin{tabular}{ll}
{\bf Program} & {\bf Function } \\
\hline
\clade		&	Extracts subtrees specified by node labels\\
\condense	&	Condenses (simplifies) trees \\
\display	&	Shows trees as graphs (\ascii{} graphics or \svg) \\
\duration	&	Convert node ages into duration \\
\distance	&	Prints distances between nodes, in various ways \\
\ed				&	Stream editor (\textit{\`{a} la} \texttt{sed} or \texttt{awk}); see also \luaed{} and \sched \\
\gen			&	Random tree generator \\
\nwindent	&	Shows Newick in indented form \\ 
\labels		&	Prints node labels \\
\luaed				&	Like \ed, but uses Lua\\
\match		&	Finds matches of a tree in another one \\
\order		&	Orders tree (preserving topology) \\
\prune		&	Removes branches based on labels \\ 
\rename		&	Changes node labels according to a mapping \\
\reroot		&	(Re)roots the tree \\
\sched				&	Like \luaed, but uses Scheme\\
\stats		&	Prints tree statistics and properties \\
\support	&	Computes bootstrap support of a tree given replicate trees \\
\topology	&	Alters branch properties, preserving topology \\
\trim		&	Trims a tree at a specified depth
\end{tabular}	
\caption{The \nutils{} and their functions}
\label{tbl_prog_list}
\end{table}

\subsection*{Citing the \nutils}

If you use the \nutils{} for published work, please cite:
\bibentry{Junier_2010}


