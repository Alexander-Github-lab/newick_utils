
\section{Displaying Trees}
\label{sct_display}

Perhaps the simplest and most common operation on a \nw{} tree is just to look
at it. The \nw{} format is not very easy to understand for us humans, so we
have to produce a graphical representation from it. This is the purpose of the
\display{} program. 

\subsection{As Text}
\label{sct_display_text}

At its simplest, \display{} just outputs a text graph. Here is a
Newick-formatted tree in file \texttt{catarrhini}:
\verbatiminput{catarrhini}
And here is the same tree shown with \display{}:
\verbatiminput{dspl1_txt.cmd}
\begin{samepage}
\verbatiminput{dspl1_txt.out}
\end{samepage}
That's pretty low-tech compared to interactive, colorful graphical displays,
but if you use the shell a lot (like I do), you may find it useful.

You can use option \texttt{-w} (``width'') to set the number of columns
available for display (the default is 80):

\verbatiminput{dspl2_txt.cmd}
\begin{samepage}
\verbatiminput{dspl2_txt.out}
\end{samepage}

\subsection[As SVG]{As Scalable Vector Graphics}
\label{sct_display_svg}

First, a disclaimer: there are dozens of tools for viewing trees out there, and
I'm not interested in competing with them. The reasons why I added \svg{}
capabilities to \display{} were:
\begin{itemize}
\item I wanted to be able to produce reasonably good graphics even if no other
tool was at hand
\item I wanted to be sure that large trees could be rendered\footnote{I have had serious problems
visualising trees of more than 1,000 leaves using some popular software I will
not name here - either it was painfully slow, or it simply crashed}
\end{itemize}
To produce \svg, pass option \texttt{-s}:
\begin{verbatim}
$ nw_display -s catarrhini > catarrhini.svg
\end{verbatim}
Now you can visualize the result using any \svg-enabled tool (all good Web browsers can do it), or convert it to another format with, say \texttt{rsvg} or Inkscape. The following \pdf{} image was produced like this:

\begin{verbatim}
$ inkscape -f catarrhini.svg -A catarrhini.pdf
\end{verbatim}

\begin{center}
 \includegraphics{dspl_svg1_svg.pdf}
\end{center}
All \svg{} images shown in this document were produced in the same way. In the
rest of this document we will usually skip the redirection into an \svg{} file
and omit the \svg{}-to-\textsc{pdf} conversion step.

The \svg{} produced by \display{} is designed to be easy to edit
in an interactive editor like Inkscape (\url{http://www.inkscape.org}): the
tree edges are in one group, and the text in another. So it is easy to change
the line properties of the edges, or the font family of the text, for example
(you can also do this from \display{} using a \css{} map, see
\ref{sct_display_svg_css}).

There are many options for \svg{} graphs. First, you can make radial trees,
using option \texttt{-r} 
\verbatiminput{dspl_sr_w450_catarrhini_svg.cmd}
You already know \texttt{-w}, except that for \svg{} the value is in pixels instead of columns. 

\begin{center}
\includegraphics{dspl_sr_w450_catarrhini_svg.pdf}
\end{center}

\subsubsection{Using CSS}
\label{sct_display_svg_css}

You can modify node style using \css. This is done by specifying a
\textit{\css{} map}, which is just a text file that says which style should be
applied to which node. If file \texttt{css.map} contains the following
\begin{quote} \verbatiminput{css.map} \end{quote} we can apply the style map to
the tree above:

\verbatiminput{nw_display_sr_w450_ccssmap_catarrhini_svg.cmd}

\begin{center}
 \includegraphics{nw_display_sr_w450_ccssmap_catarrhini_svg.pdf}
\end{center}

The syntax of the \css{} map file is as follows. Each line describes one style
and the set of nodes to which it applies. A line contains elements separated by
whitespace (whitespace between quotes does not count). The first element of the
line is the style, and it is a snippet of \css{} code. The second element is
either \texttt{Clade} or \texttt{Individual} and states whether the following
nodes are to be treated individually or as a clade. The next element(s) are
node labels and specify the nodes to which the style must be applied: if the
second element was \texttt{Clade}, the program finds the last common ancestor
of the nodes and applies the style to that node and all its descendants. If the
second element was \texttt{Individual}, then the style is only applied to the
nodes themselves.

In our example, the first line in \texttt{css.map} says that the style
\texttt{stroke:red} must be applied to the \texttt{Clade} defined by
\texttt{Macaca} and \texttt{Cercopithecus}, which consists of these two nodes,
their ancestor \texttt{Cercopithecinae}, and \texttt{Papio}. The \texttt{Clade}
keyword can be abbreviated to \texttt{C}, as in the next line. Line 2
prescribes that style \texttt{stroke:\#fa7} (an \svg{} hexadecimal color
specification) must be applied to the clade defined by \texttt{Homo} and
\texttt{Hylobates}, which consists of these two nodes, their last common
ancestor (unlabeled), and all its descendants (that is, \texttt{Homo},
\texttt{Pan}, \texttt{Gorilla}, \texttt{Pongo}, and \texttt{Hylobates}, as well
as the inner nodes \texttt{Hominini}, \texttt{Homininae} and
\texttt{Hominidae}). Line 3 says that the style \texttt{stroke:green} is to be applied to nodes \texttt{Colobus} and \texttt{Cercopithecus}, and only these nodes - not to the clade that they define. 
Line 4 says that the clade defined by \texttt{Homo} and
\texttt{Pan} should receive style \texttt{stroke-width:2;~stroke:blue} - note
that the quotes have been removed: they are not part of the style, rather they
allow us to improve readability by adding some whitespace.

The style of an inner clade overrides that of an outer clade, \textit{e.g.},
although the \texttt{Homo} - \texttt{Pan} clade is nested inside the
\texttt{Homo} - \texttt{Hylobates} clade, it has its own style (blue, wide
lines) which overrides the containing clade's style (pinkish with normal
width).

Also, note that \texttt{Individual} overrides \texttt{Clade}, which is why
\texttt{Cercopithecus} is green even though it belongs to a red clade.

The \texttt{Clade} and \texttt{Individual} keywords are not case-sensitive and
can be abbreviated - in fact only the first letter counts.

Styles can also be applied to labels. Option \texttt{-l} specifies the leaf
label style, option \texttt{-i} the inner node label style, and option
\texttt{-b} the branch length style. For example, the following tree, which was
produced using defaults, could be improved a bit:

\verbatiminput{nw_display_s_vertebrata_svg.cmd}
\begin{center}
  \includegraphics{nw_display_s_vertebrata_svg.pdf}
\end{center}

\noindent{}Let's remove the branch length labels, reduce
the vertical spacing, reduce the size of inner node labels (bootstrap values),
and write the leaf labels in italics, using a font with serifs:
\verbatiminput{nw_display_s_vertebrata_styled_svg.cmd}
\begin{center}
  \includegraphics{nw_display_s_vertebrata_styled_svg.pdf}
\end{center}
Option \texttt{-v} specifies the vertical spacing, in pixels, between two
successive leaves (the default is 40). Option \texttt{-b} sets the style of
branch labels, option \texttt{-l} sets the style of leaf labels, and option
\texttt{-i} sets the style of inner node labels. Note that we did not
\emph{discard} the branch lengths (we could do this with \topology), because
doing so would reduce the tree to a cladogram. Instead, we set their \css{}
style to \texttt{opacity:0} (\texttt{visibility:hidden} also works).

\subsection{Ornaments}

Ornaments are arbitrary snippets of \svg{} code that are displayed at specified
node positions. As for \css, this is done with a map. The ornament map has the
same syntax as the \css{} map, except that you specify \svg{} elements rather
than \css{} styles. The \texttt{Individual} keyword means that all nodes are
sport the ornament, while \texttt{Clade} means that only the clade's \lca{}
must be adorned. The ornament is translated in such a way that its (0,0)
coordinate corresponds to the position of the node.

The following file, \texttt{ornament.map}, instructs to draw a red circle with
a black border on \texttt{Homo} and \texttt{Pan}, and a cyan circle with a blue
border on the root of the \texttt{Homo} - \texttt{Hylobates} clade. The \svg{}
is enclosed in double quotes because it contains spaces - note that single
quotes are used for the values of \xml{} attributes. The ornament map is
specified with option \texttt{-o}:
\begin{quote}
 \verbatiminput{ornament.map}
\end{quote}
\verbatiminput{nw_display_sr_w450_oornmap_catarrhini_svg.cmd}
\begin{center}
 \includegraphics{nw_display_sr_w450_oornmap_catarrhini_svg.pdf}
\end{center}
Ornaments and \css{} can be combined:
\verbatiminput{nw_display_sr_w450_oornmap_ccssmap_catarrhini_svg.cmd}
\begin{center}
 \includegraphics{nw_display_sr_w450_oornmap_ccssmap_catarrhini_svg.pdf}
\end{center}

\subsubsection{Example: Mapping GC Content}

In a study of human rhinoviruses, I have produced a phylogenetic tree. I have
also computed the \gc{} content of all sequences, and mapped it into a color
map that goes from {\color{Blue} blue} (33.3\%) to {\color{Red} red} (44.5\%).
\texttt{b2r.map}. Then:

\verbatiminput{display_20_svg.cmd}
\includegraphics{display_20_svg.pdf}
\bigskip{}

\noindent{}As we can see, the high-\gc{} sequences are all found in the same main clade.

\subsubsection{Multiple SVG Trees}

Like all \nutils, \display{} can handle multiple trees, even in \svg{} mode.
The best way to do this is not evident: one can generate one file per tree (but
then we break the rule that every program is a filter and so writes to standard
output), or one can put all the trees in one \svg{} document (but then we have
to impose tiling or some other arrangement), or one can just output one \svg{}
document after another. This is what we do (this may change in the future). So
if you have many trees in document \texttt{forest.nw}, you can say:
\begin{quote}
\verb+$ nw_display -s forest.nw > forest_svg+
\end{quote}
But \texttt{forest\_svg} isn't valid \svg{} -- it is a concatenation of many \svg{} documents. You can just extract them into individual files with \texttt{csplit}:
\begin{quote}
\verb+$ csplit -sz -f tree_ -b '%02d.svg' forest_svg '/<?xml/' {*}+
\end{quote}
This will produce one \svg{} file per tree in \texttt{forest.nw}, named \texttt{tree\_01.svg}, \texttt{tree\_02.svg}, etc.

\subsection{Options not Covered}

\display{} has many options, and we will not describe them all here - all of
them are described when you pass option \texttt{-h}. They include support for
clickable images (with URLs to arbitrary Web pages), setting the scale bar
units, changing the root length, etc. 
