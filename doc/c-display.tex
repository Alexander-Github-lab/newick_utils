
\section[sct_display]{Displaying Trees}

Perhaps the simplest and most common operation on a \nw{} tree is just to look
at it. But \nw{} trees are not very intuitive for us humans, as we can quickly
see by looking \eg{} at a tree of Old World primates:

\typefile{display_12_txt.cmd}
\typefile{display_12_txt.out}


\noindent{}So we want to make a graphical representation from it. This is the
purpose of the \display{} program. 

\subsection[sct_display_text]{As Text}

At its simplest, \display{} just outputs a text graph. Here is the primates tree, shown with \display{}:

\placefigure
\typefile{display_1_txt.cmd}
\typefile{display_1_txt.out}

That's pretty low-tech compared to interactive, colorful, high-resolution
displays, but if you use the shell a lot (like I do), you may find it useful.

\subsubsection{Fixing Width}

\display{} will try to find the width of the terminal and
use that many columns when drawing the tree. If the output is not a terminal
(but, say, a file or pipe), or if for some reason the width cannot be found or
is not defined, then the program uses a default of 80 characters. You can use
option \code{-w} to override the defaults.

\page[no]
\typefile{display_2_txt.cmd}
\typefile{display_2_txt.out}


\subsubsection{Fixing Scale}

Option \code{-w} can also be used to fix the scale (in columns/branch length
units). To do so, just pass a negative number\footnote{This way it is
impossible to specify both width and scale, which would be absurd as one is
always derived from the other.}. For example, to use half a column per unit
length, do:

\page[no]
\typefile{display_20_txt.cmd}
\typefile{display_20_txt.out}


\noindent{} So a branch of length\footnote{To know the length of an edge, you
can use \distance{} \code{-m p -n} -- see \in{}[sct:distance].  } 10 (such as
the parent branch of Homo) occupies five characters (up to rounding error). And
to use two columns per unit length, one would pass \code{-w -2}, etc.

If there is more than one tree in the input, they will all be drawn at the same scale, which makes them easier to compare visually.

\subsubsection{Scale Bar}

If the tree is a phylogram, \display{} prints a scale bar. Its units can be
specified with option \code{-u}, the default is substitutions per site. To
suppress the scale bar, pass the \code{-S} switch. The scale bar can also "go
backwards" (option \code{-t}), \ie{} the scale bar's zero is aligned with the
leaves and units increase towards the root. This is handy when the units are
ages, \eg{} in millions of years ago, but it only makes much sense if the leaves
themselves are aligned. See \in{}[sct_duration] for an example.

\subsubsection{Styles}

Even lowly text graphics can have style! Option \code{-e} allows you to
change the way text graphs look:

\startalignment[center]
\starttabulate[|c|l|]
\NC Argument to \code{-e} \NC Effect \NC\NR
\HL 
\NC \code{r} \NC (raw) use just \type{|}, \type{-}, and \type{+}, and \type{=} for the root. \NC\NR
\NC \code{c} \NC (commas) corners are marked with \type{'} and \type{,} \NC\NR
\NC \code{s} \NC (slashes) corners are marked with \type{/} and \type{\textbackslash} \NC\NR
\NC \code{v} \NC (VT100) use VT100 pseudo-graphical characters \NC\NR
\stoptabulate
\stopalignment

\noindent{}Here are some examples:

\setupnarrower[middle=-2cm]
\startnarrower[middle]
%{\tfx
\bTABLE[|c|c|]
\setupTABLE[column][each][width=0.55\width]
	\bTR
		\bTD
			\typefile{display_21_txt.cmd}
			\typefile{display_21_txt.out}
		\eTD
		\bTD 
			\typefile{display_22_txt.cmd}
			\typefile{display_22_txt.out}
		\eTD
	\eTR
	\bTR
		\bTD
			\typefile{display_23_txt.cmd}
			\typefile{display_23_txt.out}
		\eTD
		\bTD 
			\typefile{display_24_txt.cmd}
			\externalfigure[display_24_txt][width=0.54\textwidth]
			This is a screenshot because I don't know how to show VT100 special characters in \ConTeXt{} :-)
		\eTD
	\eTR
\eTABLE
%}
\stopnarrower

\noindent{}By default, \display{} uses "slashes", unless overridden by
environment variable \code{NW\_DISPLAY\_TEXT\_STYLE}. This should have a
value of \code{raw}, \code{commas}, \code{slashes}, or \code{VT100}
(case is irrelevant, and actually only the first character is read, so you can
abbreviate to \code{r}, \code{c}, \code{s}, or \code{v}).

% \subsubsection{Placement of Inner Node Labels}
% 
% Option \code{-I} controls the placement of inner node labels. It takes an
% argument, which can be \code{l} (lowercase l -- towards the leaves),
% \code{m} (in the middle), or \code{r} (towards the root). The default
% behaviour is \code{l}.  Here is the above tree, with inner labels near the
% root:
% 
% \typefile{display_14_txt.cmd}
% \page[no]
% \typefile{display_14_txt.out}
% 
% 
% \subsection[sct_display_svg]{As Scalable Vector Graphics}
% 
% First, a disclaimer: there are dozens of tools for viewing trees out there, and
% I'm not interested in competing with them. The reasons why I included \svg{}
% capabilities (besides automation, etc.) were:
% \startitemize
% \item I wanted to be able to produce reasonably good graphics even if no other
% tool was at hand
% \item I wanted to be sure that large trees could be rendered\footnote{I have
% had serious problems visualising trees of more than 1,000 leaves using some
% popular software I will not name here - either it was painfully slow, or it
% simply crashed, or else the output was unreadable, incomplete, or otherwise
% unsuitable.}
% \stopitemize
% To produce \svg, pass option \code{-s}:
% \starttyping
% $ nw_display -s catarrhini > catarrhini.svg
% \stoptyping
% Now you can visualize the result using any \svg-enabled tool (all good Web
% browsers can do it), or convert it to another format with, say Inkscape
% (\from[URL:inkscape]).  The \svg{} produced by \display{} is designed
% to be easy to edit in an interactive editor (Inkscape, Adobe Illustrator,
% etc.): for example, the tree edges are in one group, and the text in another,
% so it is easy to change the line width of the edges, or the font family of the
% text (you can also do this from \display{} using a \css{} map, see
% \in{}[sct_display_svg_css]).
% 
% The following \pdf{} image was produced like this:
% 
% \starttyping
% $ inkscape -f catarrhini.svg -D -A catarrhini.pdf
% \stoptyping
% 
% \externalfigure[display_3_svg.pdf][type=pdf,align=center]
% 
% All \svg{} images shown in this tutorial were processed in the same way. In the
% rest of the document we will usually skip the redirection into an \svg{} file
% and omit the \svg{}-to-\cap{pdf} conversion step.
% 
% \subsubsection{Text-mode options}
% 
% Many options for \ascii{} trees also work for \svg{}: \code{-S} suppresses the
% scale bar\footnote{The positioning of the scale bar is a bit crude in \svg{}
% mode, especially for radial trees. This is mainly because of the "\svg{} string
% length curse", that is, the impossibility of finding out the length of a text
% string in \svg.  This means it is hard to ensure the scale bar will not overlap
% with a node label, unless one places it far away in a corner, which is what I do
% for now. An improvement to this is on my TODO list.}, and \code{-u} specifies
% its units; \code{-w} governs the tree's width (or fixes the scale if its
% argument is negative), except that for \svg{} the value is in pixels instead of
% columns; \code{-I} controls the placement of inner node labels. 
% 
% \subsubsection{Radial trees}
% 
% You can make radial trees by passing the \code{-r} switch:
% \typefile{display_4_svg.cmd}
% \startalignment[center]
% \externalfigure[display_4_svg]
% \stopalignment
% 
% \subsubsection[sct_display_svg_css]{Using CSS}
% 
% You can modify node style using Cascading Style Sheets (\css,
% (\from[URL:CSS])). This is done by specifying a \em{\css{} map},
% which is just a text file that says which style should be applied to which node.
% For example, if file \filename{css.map} contains the following
% \startnarrower
% \typefile{css.map}
% \stopnarrower we can apply the style map to
% the tree above by passing \code{-c}, which takes the name of the \css{} file
% as argument:
% 
% \typefile{display_5_svg.cmd}
% \startalignment[center]
%  \externalfigure[display_5_svg]
% \stopalignment
% 
% The syntax of the \css{} map file is as follows:
% \startitemize
% 	\item A line that starts with a \type{\#} (hash) is a comment, and will be
% 	ignored, as will be any line that contains only whitespace (space and TAB),
% 	as well as empty lines.
% 	\item Each line describes one style and the set of nodes to which it applies.
% 	A line contains elements separated by whitespace (whitespace between quotes
% 	does not count). 
% 	\item The first element of the line is the style, and it is a snippet of
% 	\css{} code. 
% 	\item The second element states whether the following nodes are to be treated
% 	individually or as a clade. It is either \code{Clade},
% 	\code{Individual}, or \code{Label} (which can be abbreviated to
% 	\code{C}, \code{I}, or \code{L}, respectively). 
% 	\item The remaining element(s) are node labels and specify the nodes to which
% 	the style must be applied: if the second element is \code{Clade}, the
% 	program finds the last common ancestor of the nodes and applies the style to
% 	that node and all its descendants. If the second element is
% 	\code{Individual}, then the style is only applied to the nodes themselves.
% 	If the second element is \code{Label}, then the style is applied to the
% 	labels.
% \stopitemize
% 
% \noindent{}In our example, \filename{css.map}:
% \startitemize
% 	\item line 2 states that the style \code{stroke:red} must be applied
% 		to the \code{Clade} defined by \code{Macaca} and
% 		\code{Cercopithecus}, which consists of these two nodes, their ancestor
% 		\code{Cercopithecinae}, and \code{Papio}. 
% 	\item Line 4 prescribes that style \code{stroke:\#fa7} (an \svg{}
% 		hexadecimal color specification) must be applied to the clade defined by
% 		\code{Homo} and \code{Hylobates}, which consists of these two nodes,
% 		their last common ancestor (unlabeled), and all its descendants (that is,
% 		\code{Homo}, \code{Pan}, \code{Gorilla}, \code{Pongo}, and
% 		\code{Hylobates}, as well as the inner nodes \code{Hominini},
% 		\code{Homininae} and \code{Hominidae}). 
% 	\item Line 6 instructs that the style \code{stroke:green} be applied
% 		individually to nodes \code{Colobus} and \code{Cercopithecus}, and only
% 		these nodes - not to the clade that they define.  
% 	\item Line 8 says that style \code{stroke-width:2;~stroke:blue} should be
% 		applied to the clade defined by \code{Homo} and \code{Pan} - note that
% 		the quotes have been removed: they are not part of the style, rather they
% 		allow us to improve readability by adding some whitespace.
% 	\item Line 10 specifies color \code{blue} for labels \code{Homo},
% 		\code{Pan},
% 		and \code{Hominini}.
% \stopitemize
% 
% The style of an inner clade overrides that of an outer clade, \eg,
% although the \code{Homo} - \code{Pan} clade is nested inside the
% \code{Homo} - \code{Hylobates} clade, it has its own style (blue, wide
% lines) which overrides the containing clade's style (pinkish with normal
% width).  Likewise, \code{Individual} overrides \code{Clade}, which is why
% \code{Cercopithecus} is green even though it belongs to a "red" clade.
% 
% \bigskip
% 
% Label styles can also be applied globally. Option \code{-l} (lowercase l)
% specifies the leaf label style, option \code{-i} the inner node label style,
% and option \code{-b} the branch length style. For example, the following tree,
% which was produced using defaults, could be improved a bit:
% 
% \typefile{display_6_svg.cmd}
% \startalignment[center]
%   \externalfigure[display_6_svg]
% \stopalignment
% 
% \noindent{}Let's remove the branch length labels, reduce
% the vertical spacing, reduce the size of inner node labels (bootstrap values),
% and write the leaf labels in italics, using a font with serifs:
% 
% \typefile{display_7_svg.cmd}
% \startalignment[center]
%   \externalfigure[display_7_svg]
% \stopalignment
% 
% \noindent{}Still not perfect, but much better. Option \code{-v} specifies the
% vertical spacing, in pixels, between two successive leaves (the default is 40).
% Option \code{-b} sets the style of branch labels, option \code{-l} sets the
% style of leaf labels, and option \code{-i} sets the style of inner node
% labels. Note that we did not {\em discard} the branch lengths (we could do
% this with \topology), because doing so would reduce the tree to a cladogram.
% Instead, we set their \css{} style to \code{opacity:0}
% (\code{visibility:hidden} also works).
% 
% What if we want to change the default style? Say we want the branches in blue,
% and two pixels wide? That's option \code{-d}:
% 
% \typefile{display_13_svg.cmd}
% \startalignment[center]
%   \externalfigure[display_13_svg]
% \stopalignment
% 
% \subsection{Ornaments}
% 
% Ornaments are arbitrary snippets of \svg{} code that are displayed at specified
% node positions. Like \css, this is done with a map. The ornament map has the
% same syntax as the \css{} map, except that you specify \svg{} elements rather
% than \css{} styles. The \code{Individual} keyword means that all nodes named
% on a given line sport the corresponding ornament, while \code{Clade} means
% that only the clade's \lca{} must be adorned. The ornament is translated in
% such a way that its (0,0) coordinate corresponds to the position of the node.
% In radial graphs, text ornaments are rotated like node labels.
% 
% The following file, \filename{ornament.map}, instructs to draw a red circle with
% a black border on \code{Homo} and \code{Pan}, and a cyan circle with a blue
% border on the root of the \code{Homo} - \code{Hylobates} clade.
% \code{Gorilla} node will be annotated with the word "plains", and
% \code{Pongo} with "Borneo" in italics\footnote{In fact, to annotate that these
% are the Western gorilla and the Borneo orang-utan, it would be simpler just to
% label the corresponding leaves accordingly, \ie, \code{Gorilla\_gorilla} and
% \code{Pongo\_pygmaeus}. But this is just an example\ldots}. The \svg{}
% is enclosed in double quotes because it contains spaces - note that single
% quotes are used for the values of \xml{} attributes. The ornament map is
% specified with option \code{-o}:
% \startnarrower
%  \typefile{ornament.map}
% \stopnarrower
% \typefile{display_8_svg.cmd}
% \startalignment[center]
%  \externalfigure[display_8_svg]
% \stopalignment
% 
% \noindent{}Ornaments and \css{} can be combined:
% 
% \typefile{display_9_svg.cmd}
% \startalignment[center]
%  \externalfigure[display_9_svg]
% \stopalignment
% 
% \subsubsection[sct:display:libxml]{\libxml}
% 
% If \libxml{} is being used (see Appendix \in{}[app:installing]), the handling of ornaments is more elaborate, in that some kinds of elements undergo special treatment. Besides positioning the ornament at the node's location and orienting it along the parent edge, which occur for all elements, the following occurs:
% \startitemize
% 	\item \code{<text>} elements are nudged a few pixels from the parent edge,
% 	to make the text more readable. They are also transformed so that the text
% 	is aligned with the node's position, on both sides of the tree (this
% 	involves an additional $180^{\circ}$  rotation on the left side of the
% 	tree).
% 	\item \code{<image>} elements are centered so that instead of having
% 	their top left corner at the node's position, they have the middle of the
% 	left side (this corresponds to vertical centering on an orthogonal tree).
% 	On the left side of the tree, they are also rotated and shifted so that
% 	they don't show upside-down.
% \stopitemize
% 
% If applicable, these transforms must be applied to each element separately.
% This means that the \svg{} snippet must be {\em parsed} (instead of just
% wrapped in a \code{<g>} element, as is the case when \libxml{} is not being
% used), and we use \libxml's \xml{} parser. 
% 
% In the following file, the orang-utan (\code{Pongo}) and hominines have
% several ornaments, which are spaced out along the radial axis so that they don't
% overlap. This is done simply by using the \code{x} attribute of texts and
% rectangles, as well as the \code{cx} attribute of circles and ellipses. Again,
% the node to be adorned lies at (0,0), $x$ values lie on the radial axis, and $y$
% values are perpendicular to the $x$ axis.
% 
% \typefile{display_17_txt.out}
% 
% This gives the following:
% 
% \typefile{display_18_svg.cmd}
% \startalignment[center]
% \externalfigure[display_18_svg]
% \stopalignment
% 
% As hinted above, \libxml{} also allows handling of images:
% 
% \typefile{img_r.map}
% 
% This gives the following (credits: all images are from Wikipedia):
% 
% \typefile{display_19_svg.cmd}
% \startalignment[center]
% \externalfigure[display_19_svg]
% \stopalignment
% 
% \subsubsection[sct_display_ornament_xpl_gc]{Example: Mapping GC Content}
% 
% In a study of human rhinoviruses, I have produced a phylogenetic tree,
% \filename{HRV\_ingrp.nw}. I have also computed the \gc{} content of the sequences,
% and mapped it into a gradient that goes from {\color[blue] blue} (33.3\%) to
% {\color[red] red} (44.5\%). I used this gradient to produce a \css{} map,
% \filename{b2r.map}:
% 
% \typefile{display_11_txt.cmd}
% \typefile{display_11_txt.out}
% 
% \noindent{}in which the \code{fill} values are hexadecimal color codes along
% the gradient. Then:
% 
% \typefile{display_10_svg.cmd}
% \startalignment[center]
% \externalfigure[display_10_svg]
% \stopalignment
% \bigskip{}
% 
% \noindent{}As we can see, the high-\gc{} sequences are all found in the same
% main clade.
% 
% \subsubsection{Multiple SVG Trees}
% 
% Like all \nutils, \display{} can handle multiple trees, even in \svg{} mode.
% The best way to do this was not evident: one can generate one file per tree (but
% then we break the rule that every program is a filter and so writes to standard
% output), or one can put all the trees in one \svg{} document (but then we have
% to impose tiling or some other arrangement), or one can just output one \svg{}
% document after another. This is what we do (this may change in the future). So
% if you have many trees in document \filename{forest.nw}, you can say:
% \startnarrower
% \type{$ nw_display -s forest.nw > forest_svg}
% \stopnarrower
% But \filename{forest\_svg} isn't valid \svg{} -- it is a concatenation of many \svg{} documents. You can just extract them into individual files with \code{csplit}:
% \startnarrower
% \type{$ csplit -sz -f tree_ -b '%02d.svg' forest_svg '/<?xml/' {*}}
% \stopnarrower
% This will produce one \svg{} file per tree in \filename{forest.nw}, named \filename{tree\_01.svg}, \filename{tree\_02.svg}, etc.
% 
% \subsubsection{\display{} \code{-s} stores its arguments}
% 
% When run in \svg{} mode, \display{} "remembers" its arguments, that is, it puts
% them in an \xml{} comment with the keyword \code{arguments}. It is then
% trivial to retrieve them:
% 
% \typefile{display_15_txt.cmd}
% \typefile{display_15_txt.out}
% 
% \noindent{}This is handy when one wants to re-use a set of options on another
% tree, especially after a while when one doesn't remember the exact values of the
% parameters, or which was the input tree, etc. 
% 
% \subsection{Options not Covered}
% 
% \display{} has many options, and we will not describe them all here - all of
% them are described when you pass option \code{-h}. They include support for
% clickable images (with URLs to arbitrary Web pages), nudging labels, changing
% the root length, etc. 
