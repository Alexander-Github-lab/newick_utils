%&context

\section[sct_labels]{Extracting Labels}


To get a list of all labels in a tree, use \labels:

\txtCmdOutput{labels_1}

The labels are printed out in \no. To get rid of internal labels,
use \code{-I}:

\txtCmdOutput{labels_2}

Likewise, you can use \code{-L} to get rid of leaf labels, and
with \code{-t} the labels are printed on a single line, separated by tabs
(here the line is folded due to lack of space).

\txtCmdOutput{labels_3}

If you just want the root's label, pass \code{-r}. In conjunction
with \clade{} (see \in{}[sct_subtrees]), this is handy to get support values of
nodes defined by their descendants. For example, the following shows the
support value of the clade defined by \id{HRV39} and \id{HRV85} in a
virus tree similar to that of \in{}[sct_display_ornament_xpl_gc]:

\txtCmdOutput{labels_5}

\subsection[sct_counting_leaves]{Counting Leaves in a Tree}

A simple application of \labels{} is a leaf count (assuming each leaf is
labeled - \nw{} does not require labels):

\txtCmdOutput{labels_4}

