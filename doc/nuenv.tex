%&context

\startenvironment nuenv

%%%%%%%%%%%%%%%%%%%%%%%%%%%%%%%%%%%%%%%%%%%%%%%%%%%%%%%%%%%%%%%%
% \defines

% Typographical tropes (not sure whether or not to keep emphasis)
\define \adhoc{{\em ad hoc}}
\define \apriori{{\em a priori}}
\define \ie{{\em i.e.}}
\define \eg{{\em e.g.}}
\define \vs{{\em vs.}}
\define \vv{{\em vice versa}}
\define \etc{etc.}

% higher-level (semantic) formatting
\define[1] \newconcept{{\em #1}}
\define[1] \foreign{{\it #1}}
\define[1] \code{{\tt #1}}
\define[1] \progname{{\tt #1}}
\define[1] \filename{{\tt #1}}
\define[1] \sciname{{\em #1}}	% Scientific name in italics
\define[1] \email{{\tt #1}}
\define[1] \id{{\tt #1}}
%\define[1] \degC{$#1\,^{\circ}{\rm C}$}

% frequent acronyms, abbreviations, aliases, etc.
\define \aids{\cap{aids}}
\define \ascii{\cap{ascii}}
\define \bincons{BinCons}  
%\define \blast{\cap{blast}}
%\define \blastx{\cap{blastx}}
%\define \blastz{\cap{blastz}}
%\define \blat{\cap{blat}}
\define \cegg{\cap{cegg}}
\define \cli{\cap{cli}}
\define \cmake{\cap{cm}ake}
%\define \cnc{\cap{cnc}}
%\define \cncs{\cap{cnc}s}
%\define \cnv{\cap{cnv}}
\define \cpu{\cap{cpu}}
\define \css{\cap{css}}
\define \dna{\cap{dna}}
\define \emboss{\cap{emboss}}
%\define \encode{\cap{encode}}
%\define \enredo{Enredo}
%\define \ensembl{Ens\cap{embl}}
\define \gc{\cap{gc}}
%\define \gerp{\cap{gerp}}
\define \gnu{\cap{gnu}}
%\define \gtf{\cap{gtf}}
%\define \gtr{\cap{gtr}}
\define \hev{\cap{hev}}
\define \hiv{\cap{hiv}}
%\define \hmm{\cap{hmm}}
\define \hrv{\cap{hrv}}
%\define \lagan{\cap{lagan}}	
\define \lca{\cap{lca}}
\define \libxml{libxml}
%\define \mlagan{\cap{mlagan}}	
\define \ml{\cap{ml}}	
\define \MP{\cap{mp}}	% \mp is already defined (-+)
%\define \multiz{\cap{multiz}}	
\define \nj{\cap{nj}}
\define \no{Newick order}
\define \nutils{Newick Utilities}
\define \nw{Newick}
%\define \orf{\cap{orf}}
%\define \pcr{\cap{pcr}}
\define \pdf{\cap{pdf}}
%\define \pecan{Pecan}
\define \phylip{\cap{phylip}}
\define \phyml{Phy\cap{ml}}
\define \picornaviridae{\sciname{Picornaviridae}} % apparently virus familiy names take italics
\define \rna{\cap{rna}}
%\define \scone{\cap{scone}}
%\define \sine{\cap{sine}}
%\define \slagan{\cap{slagan}}
%\define \snp{\cap{snp}}
%\define \sql{\cap{sql}}
\define \stderr{standard error}
\define \stdin{standard input}
\define \stdout{standard output}
\define \svg{\cap{svg}}
%\define \tba{\cap{tba}}
%\define \uce{\cap{uce}}
%\define \ucsc{\cap{ucsc}}
\define \unix{\cap{Unix}}
\define \upgma{\cap{upgma}}
%\define \utr{\cap{utr}}
\define \xml{\cap{xml}}

% names of programs
\define \clade{\progname{nw\_clade}}
\define \condense{\progname{nw\_condense}}
\define \display{\progname{nw\_display}}
\define \distance{\progname{nw\_distance}}
\define \duration{\progname{nw\_duration}}
\define \ed{\progname{nw\_ed}}
\define \gen{\progname{nw\_gen}}
\define \labels{\progname{nw\_labels}}
\define \luaed{\progname{nw\_luaed}}
\define \match{\progname{nw\_match}}
\define \nwindent{\progname{nw\_indent}} % \indent already exists
\define \order{\progname{nw\_order}}
\define \prune{\progname{nw\_prune}} 
\define \rename{\progname{nw\_rename}}
\define \reroot{\progname{nw\_reroot}}
\define \sched{\progname{nw\_sched}}
\define \stats{\progname{nw\_stats}}
\define \support{\progname{nw\_support}}
\define \topology{\progname{nw\_topology}}
\define \trim{\progname{nw\_trim}}

% This puts the command and its output, which normally end in _txt.cmd and
% _txt.out, respectively, on the same page.
\define[1] \txtCmdOutput{%
\bigskip%
\startsamepage%
\typefile{#1_txt.cmd}%
\typefile{#1_txt.out}%
\stopsamepage%
\bigskip}

% Same for SVGs
\define[1] \svgCmdOutput{%
\startsamepage%
\bigskip
\typefile{#1_svg.cmd}%
\midaligned{\externalfigure[#1_svg]}%
\bigskip
\stopsamepage}

% ... and for Newick (rare, but does occur - e.g., nw_indent)
\define[1] \nwCmdOutput{%
\bigskip%
\startsamepage%
\typefile{#1_nw.cmd}%
\typefile{#1_nw.out}%
\stopsamepage%
\bigskip}

% For SVGs without command line
\define[1] \svgFigure{%
\bigskip
\midaligned{\externalfigure[#1_svg]}%
\bigskip}


% and for text outputs without command line
\define[1] \txtFigure{%
\bigskip
\startsamepage
\typefile{#1}%
\stopsamepage
\bigskip
\bigskip}

% for TODO items
\define[1] \toDo{%
\color[red]{[TODO: #1]}%
}

%%%%%%%%%%%%%%%%%%%%%%%%%%%%%%%%%%%%%%%%%%%%%%%%%%%%%%%%%%%%%%%%
% Environments

\definestartstop[samepage][before={\vbox}]

%%%%%%%%%%%%%%%%%%%%%%%%%%%%%%%%%%%%%%%%%%%%%%%%%%%%%%%%%%%%%%%%
% Bibliography

\setuppublications[alternative=apa]
\setupbibtex[database={references},sort=author]

%%%%%%%%%%%%%%%%%%%%%%%%%%%%%%%%%%%%%%%%%%%%%%%%%%%%%%%%%%%%%%%%
% Fonts

% This is the best parametrization I've found so far (though to be
% honest I still don't understand all the details :-)). It allows Palatino as a
% main font while still allowing capital Gamma and monospace style for URLs -
% at least on Ubuntu...

\definetypeface [mainface] [rm] [serif] [palatino] [default] [features=default]
\definetypeface [mainface] [ss] [sans]  [delicious] [default] [features=default,
rscale=1.1]
\definetypeface [mainface] [tt] [mono]  [modern] [default] [features=default,
rscale=1.1]
\definetypeface [mainface] [mm] [math]  [palatino] [default] [encoding=texnansi]
% Euler fonts don't display the % sign (alas, poor Euler, and I am writing this
% in Basel).
%\definetypeface [mainface] [mm] [math]  [euler] [euler] [encoding=texnansi, rscale=1.03]

\setupbodyfont[mainface,12pt]

%%%%%%%%%%%%%%%%%%%%%%%%%%%%%%%%%%%%%%%%%%%%%%%%%%%%%%%%%%%%%%%%
% Layout

\setuppapersize[A4][A4]

\definelayout[title][backspace=1.5cm,width=18cm,
		     topspace=1cm,height=middle]

% Defaults:
% textwidth: 15cm
% leftmargin: 2.6 cm
% rightmargin: 2.6 cm
% \setuplayout[leftmargin=3.1cm,rightmargin=3.1cm,textwidth=14cm]
% \setuppagenumbering[alternative=doublesided,location=footer]
% \setupwhitespace[medium]
% \definelayout[wide][leftmarginwidth=20mm,backspace=10mm,width=16cm]

% \setupheadertexts
% [] [\setups{text:header:1}]
% [\setups{text:header:2}] []
% \startsetups text:header:1
% \getmarking[part][current]
% \quad
% \stopsetups
% \startsetups text:header:2
% \quad
% \getmarking[part][current]
% \stopsetups

%%%%%%%%%%%%%%%%%%%%%%%%%%%%%%%%%%%%%%%%%%%%%%%%%%%%%%%%%%%%%%%%
% Sectioning

\definehead[doctitle][title]
 
\defineframedtext[doctitleframe][
	width=\textwidth,
	background=screen,
	frame=off,
	topframe=on,
	%leftframe=on,
	bottomframe=on,
	%rightframe=on,
	rulethickness=0.5mm
]

\setuphead[doctitle][
	align=center,
	before={\startdoctitleframe},
	style={\switchtobodyfont[36pt]\bf},
	after={\stopdoctitleframe \blank[4*big]},
]

\setuphead[subsubsubsection][number=no,style={\switchtobodyfont[14pt]\bf}]

%%%%%%%%%%%%%%%%%%%%%%%%%%%%%%%%%%%%%%%%%%%%%%%%%%%%%%%%%%%%%%%%
% Lengths

\define \pubscale{950}	% for included publications

%%%%%%%%%%%%%%%%%%%%%%%%%%%%%%%%%%%%%%%%%%%%%%%%%%%%%%%%%%%%%%%%
% Math functions
% \newcommand{\argmax}{\mathop{\rm{arg\,max}}}

%%%%%%%%%%%%%%%%%%%%%%%%%%%%%%%%%%%%%%%%%%%%%%%%%%%%%%%%%%%%%%%%
% URLs
%
% \setupurl[style=type]
\useURL[URL:newick-format][http://evolution.genetics.washington.edu/phylip/newicktree.html]
\useURL[URL:Inkscape][http://www.inkscape.org]
\useURL[URL:CSS][www.w3.org/Style/CSS]
\useURL[URL:CEGG-NU][http://cegg.unige.ch/newick_utils]
\useURL[URL:Github][https://github.com/tjunier/newick_utils/wiki]
\useURL[URL:Hominoidea][http://en.wikipedia.org/wiki/Hominoidea]
\useURL[URL:palaeos][www.palaeos.com]
\useURL[URL:Scheme][http://www.r6rs.org]
\useURL[URL:Guile][http://www.gnu.org/software/guile]
\useURL[URL:Lua][http://www.lua.org]
\useURL[URL:Valgrind][http://www.valgrind.org]

% NOTE: I use this to wrap URLs (using '||' to mark breakpoints at slashes). On
% my Fedora box it works (i.e., it inserts a '/' when needed), but it seems to
% fail under Ubuntu.
\setuphyphenmark[sign=/]

%%%%%%%%%%%%%%%%%%%%%%%%%%%%%%%%%%%%%%%%%%%%%%%%%%%%%%%%%%%%%%%%
% Floats
\setupcaptions[style=small]

% for my own marginalia
\setupinmargin[style=\slx,align=inner,width=3.5cm,location=flushright]
% To suppress all margin notes, use:
% \define[1] \marg{}
\define[1] \marg{\margintext{#1}}

% For floating remarks
\definefloat[remark][remarks]
\setupfloat[remark][strut=yes,frame=off,bottomframe=on,leftframe=on,rightframe=on,framecorner=round,background=screen]
\setupcaption[remark][frame=on,location=top]

% For "silent" tables, i.e. w/o caption. This is useful when I don't want to
% number the table, but still want to use \placetable, e.g. to use in-float
% footnotes. But then, maybe I should number tables anyway...
\definefloat[silenttable][silenttables][table]
\setupcaption[silenttable][location=none]
%%%%%%%%%%%%%%%%%%%%%%%%%%%%%%%%%%%%%%%%%%%%%%%%%%%%%%%%%%%%%%%%
% Lists

\definedescription[definition][location=left,headstyle=bold]
\setupcombinedlist[content][alternative=b]

\setupitemize[packed]

%%%%%%%%%%%%%%%%%%%%%%%%%%%%%%%%%%%%%%%%%%%%%%%%%%%%%%%%%%%%%%%%
% Footnotes
\setupfootnotes[width=\textwidth]

%%%%%%%%%%%%%%%%%%%%%%%%%%%%%%%%%%%%%%%%%%%%%%%%%%%%%%%%%%%%%%%%
% Indenting
%\setupindenting[2em,yes]

%%%%%%%%%%%%%%%%%%%%%%%%%%%%%%%%%%%%%%%%%%%%%%%%%%%%%%%%%%%%%%%%
% Interaction
\setupcombinedlist[content][interaction=all]	% ToC
% NOTE: state=start fails with v. 2012.06.26, and causes \externalfigure to cause an emergency stop
\setupinteraction[state=start]


\stopenvironment
