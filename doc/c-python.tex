%&context

% TODO find a way of syntax-highlighting te Python code
%\lstset{
%	language=Python,
%	basicstyle=\ttfamily,
%	keywordstyle=\color{NavyBlue},
%	stringstyle=\color{OliveGreen},
%	commentstyle=\color{red},
%	frame=single,
%	frameround=tttt,
%	framexleftmargin=8mm,
%	numbers=left
%}

\chapter[chap_python_lib]{Python Bindings}


Although the \nutils{} are primarily designed for shell use, it is also possible
to use their functions from Python programs: all the core functionality of the
utilities is bundled in a C library, \filename{libnw}, which can be accessed
through Python's \filename{ctypes} module. The distribution contains a file,
\filename{newick\_utils.py}, that provides the Python to C mappings; it also
builds an object-oriented interface over it.

Let's say we want to add a utility that prints simple statistics about trees,
like the number of nodes, the depth, whether it is a cladogram or a phylogram,
etc (in other words, a Python version of \stats). We will call it
\progname{nw\_info.py}, and we'll pass it a \nw{} file on standard input, so the
usage will be something like:

\starttyping
$ nw_info.py < data/catarrhini
\stoptyping

The overall structure of this program is simple: iteratively read
all the input trees, and do something with each of them:

[TODO: syntax highlighting]
\starttyping
from newick_utils import *

for tree in Tree.parse_newick_input():
    pass # process tree here!
\stoptyping

Line 1 imports definitions from the \filename{newick\_utils.py}
module. Line 3 is the main loop: the \code{Tree.parse\_newick\_input}
reads standard input and yields an instance of class \code{Tree} for each
Newick string. We can now work with it, using methods of class \code{Tree} or adding our own:

[TODO: syntax highlighting]
\typefile{../src/nw_info.py}

When we run the program, we get:

\txtCmdOutput{python_1}

As you can see, most of the work is done by methods called on the \code{tree}
object, such as \code{get\_leaf\_count} which (surprise!) returns the number
of leaves of a tree. But since there is no method for counting polytomies, we
added our own function, \code{count\_polytomies}, which takes a \code{Tree}
object as argument.

As another example, a simple implementation of \reroot{} is found in
\progname{src/nw\_reroot.py}. It demonstrates two approaches: a heavily
object-oriented one, in which the user mainly calls methods on Python objects,
and a "thin" one, in which the calls are essentially to C functions through
\filename{libnw}. While not as fast as \reroot{}, its performance is still quite
acceptable, especially in "thin" mode.

\section{API Documentation}

Detailed information about all classes and methods available for accessing the
\nutils{} library from Python is found in file \filename{newick\_utils.py}. Note that the library must be installed on your system, which means that you must compile from source.
