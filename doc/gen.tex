\section{Generating Random Trees}
\label{sct_gen}

\gen{} generates clock-like random trees, with exponentially distributed branch
lengths. Nodes are sequentially labeled.

\verbatiminput{gen_1_svg.cmd}
\begin{center}
\includegraphics{gen_1_svg.pdf}
\end{center}

\noindent{}Here I pass option \texttt{-s}, whose argument is the pseudorandom
number generator's seed, so that I get the same tree every time I produce this
document. Normally, you not use it if you want a different tree every time.
Other options are \texttt{-d}, which specifies the tree's depth, and
\texttt{-l}, which sets the average branch length.

I use random trees to test the other applications, and also as a kind of null
model to test to what extent a real tree departs from the null hypothesis.
