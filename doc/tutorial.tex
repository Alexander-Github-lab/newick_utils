\documentclass[a4paper,10pt]{report}

\usepackage{url}
\usepackage{graphics}
\usepackage{verbatim}

% Title Page
\title{Newick Utilities Tutorial}
\author{Thomas Junier}


\newcommand{\nutils}{Newick Utilities}
\newcommand{\unix}{\textsc{Unix}}
\newcommand{\ascii}{\textsc{ASCII}}
\newcommand{\svg}{\textsc{SVG}}
\newcommand{\stdin}{standard input}
\newcommand{\nw}{Newick}
\newcommand{\css}{\textsc{CSS}}

% names of programs
\newcommand{\display}{\texttt{nw\_display}}
\newcommand{\reroot}{\texttt{nw\_reroot}}
\newcommand{\topology}{\texttt{nw\_topology}}

\begin{document}
\maketitle
\tableofcontents

\chapter{Introduction}

The \nutils{} are a set of \unix{} shell programs for working with Newick-formatted phylogenetic trees. Their main features are:
\begin{itemize}
 \item they require no user interaction.\footnote{Why this is a good thing is not the focus of this document: I shall assume that if you are reading this, you already know when a command-line interface is better than an interactive interface.}
 \item they can work on any number of trees at a time\footnote{Strictly speaking, a few applications are limited to one input tree because working on more than one is not practical}
 \item they perform reasonably well with large trees
\end{itemize}
They are not tools for \emph{making} phylogenies. Rather, they are for working with existing ones, by which I mean manipulating the tree or extracting information from it: rerooting, simplifying, extracting subtrees, printing branch lengths and distances, etc - a glance at the table of contents of this document should give you an idea.

Each of the programs performs one task (with some variants). For example, here is how you would reroot a series of phylograms contained in file \texttt{mytrees.nw} using node \texttt{Dmelano} as outgroup:

\begin{verbatim}
$ nw_reroot mytrees.nw Dmelano
\end{verbatim} 
Now, you might want to make cladograms from the rerooted trees. This would be done like this:
\begin{verbatim}
$ nw_reroot mytrees.nw Dmelano | nw_topology -
\end{verbatim}
As you can see, the rerooted tree from \reroot{} is just piped into \topology, which takes care of removing branch lengths, in a typical shell pipeline.

This document is organized as follows: chapter \ref{chap_general} discusses common features of the \nutils, chapter \ref{chap_simple} shows examples of simple tasks, and chapter \ref{chap_adv} has examples of more advanced tasks. All the examples shown in this document can be reproduced using files in the \texttt{data} directory.

\section{Conventions}

Most of the utilities produce Newick output. This is a rather terse format, not easily readable for humans. Therefore, I will generally show the result in graph form, either as \ascii{} or as (rendered!) \svg. The conversion from Newick to graph will be performed with \display{}. To avoid tedious repetitions, however, I will not always explicitly show the \display{} step; that is, if I say that 
\begin{verbatim}
$ nw_clade catarrhini.nw Pongo Pan
\end{verbatim} 
results in this:
\begin{samepage}
\begin{verbatim}
                  +-----------------+ Gorilla     
                  |                               
 +----------------+ Homininae +----------+ Pan    
 |                +-----------+ Hominini          
=| Hominidae                  +----------+ Homo   
 |                                                
 +---------------------------------+ Pongo  
\end{verbatim} 
\end{samepage}
it will mean that the last step was \display{}, something like this:
\begin{verbatim}
$ nw_clade catarrhini.nw Pongo Pan | nw_display -
\end{verbatim}
If the graph is \svg, then it will have been \display{} \verb+-s+. See section \ref{sct_display} for more on displaying trees.

\chapter{General Remarks}
\label{chap_general}

The following applies to all programs in the \nutils{} package.

\section{Help}
\label{sect_help}

All programs print a help message if passed option \texttt{-h}:

\begin{samepage}
\begin{verbatim}
$ nw_indent -h
Indents the Newick, making structure more clear.

Synopsis
--------

nw_indent [-cht:] <newick trees filename|->

Input
-----

Argument is the name of a file that contains Newick trees, or '-' (in
which case trees are read from standard input).

Output
------

By default, prints the input tree, with each parenthesis and each leaf on a
line of its own, and indented a multiple of '  ' (two spaces) to reflect
structure. The default output is valid Newick.
[...]
\end{verbatim}
\end{samepage}
The help page describes the program's purpose, its input and output, and its options, in a format reminiscent of \unix{} manpages. It also shows a few examples. All examples can be tried out using files in the \texttt{data} directory.

\section{Input}
\label{sect_input}

Since the \nutils{} are for working with trees, it should not be a surprise that all programs take Newick as input. The Newick format is a file format for (phylogenetic) trees. It is one of the most widely used and uderstood formats.
A complete description can be found at \url{http://evolution.genetics.washington.edu/phylip/newicktree.html}.

The input tree(s) are always the first argument to the program (after any options). They may either be stored in a file, or piped on \stdin{}. In the latter case, the filename is replaced by a '\texttt{-}' (dash):

\begin{samepage}
\begin{verbatim}
$ nw_display mytrees.nw
\end{verbatim}
is the same as
\begin{verbatim}
$ cat mytrees.nw | nw_display -
\end{verbatim}
\end{samepage}
Of course the second form is only really useful when chaining several programs into pipelines.

\subsection{Multiple Trees in Input}

Most programs can work on any number of trees. That is, in the example above the file \texttt{mytrees.nw} may contain one or more trees, preferably one tree per line\footnote{It may also work if the trees are formatted differently, but the programs were only tested on input files with one tree per line.}. The task will be performed on each tree in the input. So if you need to reroot 1,000 trees on the same outgroup, you can do it all in a single step (see \ref{sct_reroot}).

\section{Output}
\label{sect_output}

The \nutils{} either modify trees and print the result, or print information about the trees. In the first case, the output is also Newick.

\chapter{Simple Tasks}
\label{chap_simple}

\section{Displaying Trees}
\label{sct_display}

Perhaps the simplest and most common operation on a \nw{} tree is just to look at it. The \nw{} format is not very easy to understand for us humans, so we have to produce a graphical representation from it. This is the purpose of the \display{} program. 

\subsection{As Text}
\label{sct_display_text}

At the simplest, \display{} just outputs a text graph:

\verbatiminput{dspl1_txt.cmd}
\verbatiminput{dspl1_txt.out}

That's pretty low-tech compared to interactive, colorful graphical displays, but if you use the shell a lot (like I do), you may find it useful.

You can use option \texttt{-w} (``width'') to set the number of columns available for display:

\verbatiminput{dspl2_txt.cmd}
\verbatiminput{dspl2_txt.out}

\subsection{As \svg}
\label{sct_display_svg}

Option \texttt{-s} produces \svg{} output:

\begin{verbatim}
$ nw_display -s catarrhini > catarrhini.svg
\end{verbatim}

Now you can visualize the result using any \svg-enabled tool (all good Web browsers can do it), or convert it to another format with, say \texttt{rsvg} or Inkscape. The following image was produced like this:

\begin{verbatim}
$ inkscape -f catarrhini.svg -A catarrhini.pdf
\end{verbatim}

\begin{center}
 \includegraphics{dspl_svg1_svg.pdf}
\end{center}

There are many options for \svg{} graphs.First, you can make radial trees, using option \texttt{-r} (from now on I will skip the redirection into an \svg{} file):

\begin{verbatim}
$ nw_display -sr -w 450 catarrhini
\end{verbatim}
\verbatiminput{dspl_sr_w450_catarrhini_svg.cmd}

You already know \texttt{-w}, except that for \svg{} the value is in pixels instead of columns. 

\begin{center}
 \includegraphics{dspl_sr_w450_catarrhini_svg.pdf}
\end{center}

\subsubsection{Using \css}
\label{sct_display_svg_css}

You can modify node style using \css. This is done by specifying a \textit{\css{} map}, which is just a text file that says which style should be applied to which node. If file \texttt{css.map} contains the following

\verbatiminput{css.map}

we can apply the style map to the tree above:

\verbatiminput{nw_display_sr_w450_ccssmap_catarrhini_svg.cmd}

\begin{center}
 \includegraphics{nw_display_sr_w450_ccssmap_catarrhini_svg.pdf}
\end{center}

The syntax of the \css{} map is as follows. Each line describes one style and the set of nodes to which it applies.

\section{Indenting}
\label{sct_indent}

\section{Rooting and Rerooting}
\label{sct_reroot}

Rooting transforms an unrooted tree into a rooted one, and rerooting changes a rooted tree's root. Some tree-building methods produce rooted trees (e.g., \textsc{UPGMA}), others produce unrooted ones (neighbor-joining, maximum-likelihood). 

The Newick format is implicitly rooted, in the sense that there is a 'top' node from which all other nodes descend. Some applications regard a tree with a trifuraction at the top node as unrooted. 

One way of (re)rooting a tree is to specify an \textit{outgroup}. In the simplest case, this is a single leaf. The root is then placed in such a way that one of its children is the outgroup, while the other child is the rest of the tree (sometimes known as the \textit{ingroup}).

\section{Labels}

\chapter{Advanced Tasks}
\label{chap_adv}
\end{document}
