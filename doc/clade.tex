

\section{Extracting Subtrees}
\label{sct_subtrees}

You can extract a clade (\texttt{aka} subtree) from a tree with \clade. As
usual, a clade is specified by a number of node labels, of which the program
finds the last common ancestor, which unequivocally determines the clade (see
Appendix \ref{sct_def_clades}).  We'll use the catarrhinian tree again for
these examples:

\verbatiminput{clade_2_svg.cmd}
\begin{center}
\includegraphics{clade_2_svg.pdf}
\end{center}

In the simplest case, the clade you want to extract has its own, unique label.
This is the case of \texttt{Cercopithecidae}, so you can extract the whole
cercopithecid subtree (Old World monkeys) using just that label:

\verbatiminput{clade_1_svg.cmd}
\begin{center}
  \includegraphics{clade_1_svg.pdf}
\end{center}

Now suppose I want to extract the apes subtree. These are the Hominidae
("great apes") plus the gibbons (\textit{Hylobates}). But the corresponding
node is unlabeled in our tree (it should be \texttt{Hominoidea}), so we need to specify (at least) two descendants:

\verbatiminput{clade_3_svg.cmd}
\begin{center}
  \includegraphics{clade_3_svg.pdf}
\end{center}

\noindent{}The descendants do not have to be leaves: here I use \texttt{Hominidae}, an inner node, and the result is the same.

\verbatiminput{clade_4_svg.cmd}
\begin{center}
  \includegraphics{clade_4_svg.pdf}
\end{center}


