

\section{Extracting Subtrees}
\label{sct_subtrees}

You can extract a clade (\texttt{aka} subtree) from a tree with \clade. As
usual, a clade is specified by a number of node labels, of which the program
finds the last common ancestor, which unequivocally determines the clade (see
Appendix \ref{sct_def_clades}).  We'll use the catarrhinian tree again for
these examples:

\verbatiminput{clade_2_svg.cmd}
\begin{center}
\includegraphics{clade_2_svg.pdf}
\end{center}

In the simplest case, the clade you want to extract has its own, unique label.
This is the case of \texttt{Cercopithecidae}, so you can extract the whole
cercopithecid subtree (Old World monkeys) using just that label:

\verbatiminput{clade_1_svg.cmd}
\begin{center}
  \includegraphics{clade_1_svg.pdf}
\end{center}

Now suppose I want to extract the apes subtree. These are the Hominidae
("great apes") plus the gibbons (\textit{Hylobates}). But the corresponding
node is unlabeled in our tree (it should be \texttt{Hominoidea}), so we need to specify (at least) two descendants:

\verbatiminput{clade_3_svg.cmd}
\begin{center}
  \includegraphics{clade_3_svg.pdf}
\end{center}

\noindent{}The descendants do not have to be leaves: here I use \texttt{Hominidae}, an inner node, and the result is the same.

\verbatiminput{clade_4_svg.cmd}
\begin{center}
  \includegraphics{clade_4_svg.pdf}
\end{center}

\subsection{Monophyly}

You can check if a set of leaves\footnote{In future versions I
may extend this to inner nodes} form a monophyletic group by passing option
\texttt{-m}: \clade{} will report the subtree only if it has no descendant leaf other than those specified.  For example, we can ask if the African apes (humans, chimp, gorilla) form a monophyletic group:

\verbatiminput{clade_7_svg.cmd}
\begin{center}
  \includegraphics{clade_7_svg.pdf}
\end{center}

\noindent{}Yes, they do -- it's subfamily Homininae. On the other hand, the Asian apes (orangutan and gibbon) do not:

\verbatiminput{clade_8_txt.cmd}
\verb+[no output]+ \\

\noindent{}Maybe early hominines split from orangs in South Asia before moving to Africa.

\subsection{Context}

You can ask for $n$ levels above the clade by passing option \texttt{-c}: 

\verbatiminput{clade_5_svg.cmd}
\begin{center}
  \includegraphics{clade_5_svg.pdf}
\end{center}

\noindent{}In this case, \clade{} computed the \lca{} of \texttt{Gorilla} and
\texttt{Homo}, "climbed up" two levels, and output the subtree at that point.
This is useful when you want to extract a clade with its nearest neighbor(s). I
use this when I have several trees in a file and my clade's nearest neighbors
aren't always the same.

\subsection{Siblings}

You can also ask for the siblings of the specified clade. What, for example, is
the sister clade of the cercopithecids? Ask for \texttt{Cercopithecidae} and
pass option \texttt{-s}:

\verbatiminput{clade_6_svg.cmd}
\begin{center}
  \includegraphics{clade_6_svg.pdf}
\end{center}

\noindent{}Why, it's the good old apes, of course. I use this a lot when I
want to get rid of the outgroup: specify the outgroup and pass \texttt{-s} --
behold!, you have the ingroup.

Finally, although we are usually dealing with bifurcating trees, \texttt{-s}
also applies to multifurcations: if a node has more than one sibling, \clade{} reports them all, in \nw{} order.

\subsection{Limits}

\clade{} assumes that node labels are unique. This should change in the future.
