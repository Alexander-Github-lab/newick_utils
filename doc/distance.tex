\section{Extracting Distances}

There are many ways of defining distances in a tree. First, we have to consider
whether the tree is a cladogram or a phylogram: in the former case, the
distances will be measured in whatever units the tree was built with (usually
substitutions per site), in the latter, it will just be the number of ancestors.

Then we can measure distance in a variety of ways. This is the job of
\distance. By default, it prints the distance from the root of the tree to all
labeled leaves, in the order in which they appear in the Newick. Let's look at
distances in the monkeys tree:

\begin{center}
\includegraphics{dspl_svg1_svg.pdf}
\end{center}

\verbatiminput{dist_1_txt.cmd}
\begin{samepage}
\verbatiminput{dist_1_txt.out}
\end{samepage}
This means that the distance from the root to \texttt{Gorilla} is 56, etc. Option \texttt{-n} shows the labels:
\verbatiminput{dist_2_txt.cmd}
\begin{samepage}
\verbatiminput{dist_2_txt.out}
\end{samepage}

There are two main parameters to \distance: the \emph{method} and the
\emph{selection}. The method determines how to compute the distance (from what
node to what node), and the selection determines for which nodes the program is
to compute distances. Let's look at examples.

\subsection{Selection}

In this section we will show the different selections, using the default
distance method. The selection is set by option \texttt{-s}. The argument to
this option is the selection type (see melow), it can be abbreviated to a
single letter (case is not significant). To illustrate the selection, we need a tree that has both labeled and unlabeled leaves and inner nodes:

\begin{center}
\includegraphics{dist_3_svg.pdf}
\end{center}

\subsubsection{All labeled Leaves}

The selection consists of all leaves with a label. This is the default, as
leaves will mostly be labeled and we're generally more interested in leaves
than inner nodes.

\verbatiminput{dist_4_txt.cmd}
\verbatiminput{dist_4_txt.out}

\subsubsection{All leaves}

Option \texttt{-s l}, \texttt{-s leaves}, etc. This takes all leaves into account, whether they are labeled or not.

\subsection{Methods}

In this section we will take the default selection and vary the method.

\subsubsection{Distance from the Tree's Root}

This is the default method: for each node in the selection, the program prints
the distance from the tree's root to the node. This was shown in the example above, so I won't repeat it here.




