\section{Extracting Distances}

There are many ways of defining distances in a tree. First, we have to consider
whether the tree is a cladogram or a phylogram: in the former case, the
distances will be measured in whatever units the tree was built with (usually
substitutions per site), in the latter, it will just be the number of ancestors.

Then we can measure distance in a variety of ways. This is the job of
\distance. By default, it prints the distance from the root of the tree to all
labeled leaves, in the order in which they appear in the Newick. Let's look at
distances in the monkeys tree:

\begin{center}
\includegraphics{dspl_svg1_svg.pdf}
\end{center}

\verbatiminput{dist_1_txt.cmd}
\begin{samepage}
\verbatiminput{dist_1_txt.out}
\end{samepage}
This means that the distance from the root to \texttt{Gorilla} is 56, etc. Option \texttt{-n} shows the labels:
\verbatiminput{dist_2_txt.cmd}
\begin{samepage}
\verbatiminput{dist_2_txt.out}
\end{samepage}




