\section{Condensing}
\label{sct_condense}

Condensing a tree means reducing its size in a systematic, meaningful way
(compare this to \emph{pruning} (\ref{sct_prune}) which arbitrarily removes
branches). Currently the only condensing method available is simplifying
clades in which all leaves have the same label - perhaps because they belong
to the same taxon, etc.. Consider this tree:

\includegraphics{condense_1_svg.pdf}

it has a clade that consists only of A, another of only C, plus a B leaf.
Condensation will replace those clades by an A and a C leaf, respectively:

\verbatiminput{condense_2_svg.cmd}
\includegraphics{condense_2_svg.pdf}

Now the A and B leaves stand for whole clades.  A typical use of this is
producing genus trees from species trees, or any higher-level tree from a
lower-level one - see \ref{sct_higher_lvl}).
