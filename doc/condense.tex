\section{Condensing}
\label{sct_condense}

Condensing a tree means reducing its size in a systematic, meaningful way
(compare this to \emph{pruning} (\ref{sct_prune}) which arbitrarily removes
branches, and to \textit{trimming} (\ref{sct_trim}) which cuts a tree at a
specified depth). Currently the only condensing method available is
simplifying clades in which all leaves have the same label - for example
because they belong to the same taxon, etc. Consider this tree:

\begin{center}
\includegraphics{condense_1_svg.pdf}
\end{center}

\noindent{}it has a clade that consists only of A, another of only C, plus a B
leaf.  Condensing will replace those clades by an A and a C leaf,
respectively:

\verbatiminput{condense_2_svg.cmd}
\begin{center}
\includegraphics{condense_2_svg.pdf}
\end{center}

\noindent{}Now the A and B leaves stand for whole clades. The tree is simpler,
but the information about the relationships of A, B and C is conserved, while
the details of the A and C clades is not.  A typical use of this is producing
genus trees from species trees (or any higher-level tree from a lower-level
one), or checking consistency with other data: For example condensing the virus
tree of section \ref{sct_higher_rank} gives this:

\begin{center}
%\includegraphics{condense_3_svg.pdf} TODO: reenable
\end{center}

\noindent{}The relationships between the species is now evident -- as is the
fact that the various isolates do cluster within species in the first place.
This need not be the case, and renaming-then-condensing is a useful technique
for checking this kind of consistency in a tree (see
\ref{sct_check_consistency} for more examples).
