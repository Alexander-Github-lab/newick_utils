\section{Ordering Nodes}
\label{sct_ordering}

Ordering nodes means changing the order of a nodes's children (without altering
the tree's topology). The reason for doing this can be to check if 
different \nw{} strings represent different trees, or to ensure that
they do. Program \order{} performs this task.

For example, files \texttt{falconiformes} and \texttt{falconiformes\_2} are
different, and they yield different graphs:

\verbatiminput{order_1_svg.cmd}
\begin{center}
\includegraphics{order_1_svg.pdf}
\end{center}

\verbatiminput{order_2_svg.cmd}
\begin{center}
\includegraphics{order_2_svg.pdf}
\end{center}

\noindent{}But do they represent different phylogenies? In other words, do they
differ by more than just the ordering of nodes? To check this, we pass them to \order:

\verbatiminput{order_2_svg.cmd}
