%&latex

\section{Ordering Nodes}
\label{sct_order}

Two trees that differ only by the order of children nodes within the parent
convey the same biological information, even if the text (\nw) and graphical
representations differ.  For example, files \texttt{falconiformes\_1} and
\texttt{falconiformes\_2} are different, and they yield different images:

\verbatiminput{order_1_svg.cmd}
\begin{center}
\includegraphics{order_1_svg.pdf}
\end{center}

\verbatiminput{order_2_svg.cmd}
\begin{center}
\includegraphics{order_2_svg.pdf}
\end{center}

\noindent{}But do they represent different phylogenies? In other words, do they
differ by more than just the ordering of nodes? To check this, we pass them to
\order{} and use \texttt{diff} to compare the results\footnote{One could also compute a checksum using \texttt{md5sum}, etc}:

\verbatiminput{order_3_txt.cmd}
\verbatiminput{order_3_txt.out}

So, after ordering, the trees are the same: they tell the same biological
story. Note that these trees are cladograms. If you have trees with branch
lengths, this approach will only work if the lengths are identical, which may or
may not be what you want. You can get rid of the branch lengths using
\topology{} (see \ref{sct_topology}).

\subsection{Variants}

Other ordering criteria are available through option \texttt{-c}. To order a
tree by number of descendants (\textit{i.e.}, "light" nodes before "heavy"
nodes), pass \texttt{-c n}. This has the effect of "ladderizing" trees which are
heavily imbalanced. Consider this tree:

\begin{center}
\includegraphics{order_4_svg.pdf}
\end{center}

\noindent{}Here is the same tree, reordered by number of descendants: light
nodes appear before (clockwise) heavy nodes:

\verbatiminput{order_5_svg.cmd}
\begin{center}
\includegraphics{order_5_svg.pdf}
\end{center}

\subsubsection{De-ladderizing}

Incidentally, "ladderizing" a tree may not be a good idea, because it lends
itself to misinterpretations. For example, the following tree leads some people
(including professional biologists, apparently \cite{Baum_2005}) to the
following mistakes:

\begin{center}
\includegraphics{order_6_svg.pdf}
\end{center}
\begin{itemize}
	\item there is a "chain of being" with "higher" and "lower" organisms,
	with (surprise!) humans at the top; "higher" can be interpreted in
	various ways, including "more perfect", or "more evolved" or even
	morally superior. This is known as the \textit{scala natur\ae} fallacy.
	\item there is a "main line" that progressively leads to (surprise!)
	humans, with "offshoots" along the way -- lowly lampreys branching out
	first, then sharks, etc. 
	\item early-branching species (this is itself an error) are
	"primitive": in our case, it would mean that the last common ancestor
	of lampreys and humans was a lamprey (or very like one); that the
	\lca{} of humans and sharks was
	very much like a modern shark, etc.
\end{itemize}
For a comprehensive discussion of errors in tree-thinking, see
\cite{Gregory_2008}.  Now, to prevent these errors, one can reorder the tree in
such a way as to remove the ladder. This is done by passing \texttt{-c d}. The
tree is topologically identical, so it tells the same biological story:

\verbatiminput{order_7_svg.cmd}
\begin{center}
\includegraphics{order_7_svg.pdf}
\end{center}

\noindent{}It is less easy now to construe that there is a chain of being, or
that evolution is progressive, etc. Unfortunately, some folks take the new tree
to mean that humans are more closely related to amphibians (\textit{Xenopus})
than to birds (\textit{Columba}). There is no substitute to actually learn how
to interpret trees, I'm afraid. 
