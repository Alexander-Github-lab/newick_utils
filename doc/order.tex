\section{Ordering Nodes}
\label{sct_order}

Two trees that differ only by the order of children nodes within the parent
convey the same biological information, even if the text (\nw) and graphical
representations differ.  For example, files \texttt{falconiformes\_1} and
\texttt{falconiformes\_2} are different, and they yield different images:

\verbatiminput{order_1_svg.cmd}
\begin{center}
\includegraphics{order_1_svg.pdf}
\end{center}

\verbatiminput{order_2_svg.cmd}
\begin{center}
\includegraphics{order_2_svg.pdf}
\end{center}

\noindent{}But do they represent different phylogenies? In other words, do they
differ by more than just the ordering of nodes? To check this, we pass them to
\order{} and use \texttt{diff} to compare the results\footnote{One could also compute a checksum using \texttt{md5sum}, etc}:

\verbatiminput{order_3_txt.cmd}
\verbatiminput{order_3_txt.out}

So, after ordering, the trees are the same: they tell the same biological
story. Note that these trees are cladograms. If you have trees with branch
lengths, this apprach will only work if the lengths are identical, which may or
may not be what you want. You can get rid of the branch lengths using
\topology{} (see \ref{sct_topology}).
