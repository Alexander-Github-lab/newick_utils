

\section{Extracting Labels}
\label{sct_labels}

To get a list of all labels in a tree, use \labels:

\begin{samepage}
\verbatiminput{labels_1_txt.cmd}
\verbatiminput{labels_1_txt.out}
\end{samepage}

\noindent{}The labels are printed out in \no. To get rid of internal labels,
use \texttt{-I}:

\begin{samepage}
\verbatiminput{labels_2_txt.cmd}
\verbatiminput{labels_2_txt.out}
\end{samepage}

\noindent{}Likewise, you can use \texttt{-L} to get rid of leaf labels, and
with \texttt{-t} the labels are printed on a single line, separated by tabs
(here the line is folded due to lack of space).

\begin{samepage}
\verbatiminput{labels_3_txt.cmd}
\verbatiminput{labels_3_txt.out}
\end{samepage}

\noindent{}If you just want the root's label, pass \texttt{-r}. In conjunction
with \clade{} (see \ref{sct_subtrees}), this is handy to get support values of
nodes defined by their descendants. For example, the following shows the
support value of the clade defined by \texttt{HRV39} and \texttt{HRV85} in a
virus tree similar to that of \ref{sct_display_ornament_xpl_gc}:

\begin{samepage}
\verbatiminput{labels_5_txt.cmd}
\verbatiminput{labels_5_txt.out}
\end{samepage}

\subsection{Counting Leaves in a Tree}
\label{sct_counting_leaves}

A simple application of \labels{} is a leaf count (assuming each leaf is
labeled - \nw{} does not require labels):

\begin{samepage}
\verbatiminput{labels_4_txt.cmd}
\verbatiminput{labels_4_txt.out}
\end{samepage}
