

\section{Extracting Labels}
\label{sct_labels}

To get a list of all labels in a tree, use \labels:

\verbatiminput{labels_xpl1_txt.cmd}
\verbatiminput{labels_xpl1_txt.out}

\noindent{}The labels are printed out in post-order. To get rid of internal labels, use \texttt{-I}:

\verbatiminput{labels_xpl2_txt.cmd}
\verbatiminput{labels_xpl2_txt.out}

\noindent{}Likewise, you can use \texttt{-L} to get rid of leaf labels (not shown). Finally, if you use \texttt{-t} the labels are printed on a single line, separated by tabs (for some reason the tabs show as spaces here, I think this is an effect of \LaTeX{}'s \verb+\verbatiminput{}+):

\verbatiminput{labels_xpl3_txt.cmd}
\verbatiminput{labels_xpl3_txt.out}

\subsection{Counting Leaves in a Tree}
\label{sct_counting_leaves}

A simple application of \labels{} is a leaf count (assuming each leaf is
labeled - \nw{} does not require labels):

\verbatiminput{labels_xpl4_txt.cmd}
\verbatiminput{labels_xpl4_txt.out}

