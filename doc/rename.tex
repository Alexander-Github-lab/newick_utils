\section{Renaming nodes}
\label{sct_rename}

Renaming nodes is the rather boring operation of changing a node's label. It
can be done \eg{} for the following reasons:
\begin{itemize}
	\item building a higher-level tree (\ie, a families tree from a tree of genera, etc)
	\item mapping one namespace into another (see \ref{sct_10_char_limit})
	\item correcting wrong names
\end{itemize}

\noindent{}Renaming is done with \rename. This takes a \emph{renaming map},
which is just a text file with the old and new names on the same line.

\subsection{Breaking the 10-character limit in \phylip{} alignments}
\label{sct_10_char_limit}

A technical hurdle with phylogenies is that some programs do not accept names
longer than, say, 10 characters in the \phylip{} alignment. But of
course, many scientific names or sequence \textsc{ID}s are longer than that.
One solution is to rename the sequences, before constructing the tree, using a
numerical scheme, \eg{}, \textit{Strongylocentrotus purpuratus} $\rightarrow$
\texttt{ID\_01}, etc. This means we have an alignment of the following form:
\begin{verbatim}
 154 259
ID_01     PTTSNSAPAL DAAETGHTSG ...
ID_02     SVSSHSVPAL DAAETGHTSS ...
...
\end{verbatim}
together with a renaming map, \texttt{id2longname.map}:
\begin{verbatim}
ID_01	Strongylocentrotus_purpuratus
ID_02	Harpagofututor_volsellorhinus
...
\end{verbatim}
The alignment's \textsc{ID}s are now sufficiently short, and we can use it to
make a tree. It will look something like this:

\verbatiminput{rename_1_svg.cmd}
\begin{center}
\includegraphics{rename_1_svg.pdf}
\end{center}

\noindent{}Not very informative, huh? But we can put back the original, long
names :

\verbatiminput{rename_2_svg.cmd}
(option \texttt{-W} specifies the mean width of label characters, in
pixels -- use it when the default is wrong, as in this case with very long
labels and small characters)
\begin{center}
\includegraphics{rename_2_svg.pdf}
\end{center}

\noindent{}Now that's better\ldots although exactly what these critters are
might not be evident. Not to worry, I've made another map and I can rename the
tree a second time on the fly:

\verbatiminput{rename_3_svg.cmd}
\begin{center}
\includegraphics{rename_3_svg.pdf}
\end{center}

\subsection{Higher-rank trees}
\label{sct_higher_rank}

Here is a tree of a few dozen enterovirus and rhinovirus isolates. I show it as
a cladogram (using \topology, see \ref{sct_topology}) because branch lengths do
not matter here. I know that these isolates belong to three species in two
genera: human rhinovirus A (\texttt{hrv-a}), human rhinovirus B
(\texttt{hrv-b}, and enterovirus (\texttt{hev}). 

\verbatiminput{rename_4_svg.cmd}
\begin{center}
\includegraphics{rename_4_svg.pdf}
\end{center}

\noindent{}I want to see if the tree correctly groups isolates of the same
species together. So I use a renaming map that maps an isolate name to its
species (note by the way that the map file can have comment, whitespace-only
and empty lines (which are all ignored), just like \textsc{css} maps (see
\ref{sct_display_svg_css}):
\begin{verbatim}
# These species belong to HRV-A
HRV16 HRV-A
HRV1B HRV-A
...
# HRV-B
HRV37 HRV-B
HRV14 HRV-B
...
# Enterovirus
POLIO1A HEV
COXA17 HEV
\end{verbatim}

\verbatiminput{rename_5_svg.cmd}
\begin{center}
\includegraphics{rename_5_svg.pdf}
\end{center}

\noindent{}As we can see, it does. This would be even better if we could
somehow simplify the tree so that clades of the same species were reduced to a
single leaf. And, that's exactly what \condense{} does (see below).
