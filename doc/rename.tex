\section{Renaming nodes}
\label{sct_rename}

Renaming nodes is the rather boring operation of changing a node's label. It
can be done \eg{} for the following reasons:
\begin{itemize}
	\item building a higher-level tree (\ie, a families tree from a tree of genera, etc)
	\item mapping one namespace into another
	\item correcting wrong names
\end{itemize}

Renaming is done with \rename. This takes a \emph{renaming map}, which is just
a text file with the old and new names on the same line.

\subsection{Breaking the 10-character limit in \phylip{} alignments}

A technical hurdle with phylogenies is that some programs do not accept names
longer than, say, 10 characters in the \phylip{} alignment. But of
course, many scientific names or sequence \textsc{ID}s are longer than that.
One solution is to rename the sequences, before constructing the tree, using a
numerical scheme, \eg{}, \textit{Strongylocentrotus purpuratus} $\rightarrow$
\texttt{ID\_01}, etc. This means we have an alignment of the following form:
\begin{verbatim}
 154 259
ID_01     PTTSNSAPAL DAAETGHTSG ...
ID_02     SVSSHSVPAL DAAETGHTSS ...
...
\end{verbatim}
together with a renaming map, \texttt{id2longname.map}:
\begin{verbatim}
ID_01	Strongylocentrotus_purpuratus
ID_02	Harpagofututor_volsellorhinus
...
\end{verbatim}
The alignment's \textsc{ID}s are now sufficiently short, and we can use it to
make a tree. It will look something like this:

\verbatiminput{rename_1_svg.cmd}
\begin{center}
\includegraphics{rename_1_svg.pdf}
\end{center}

\noindent{}Not very informative, huh? But we can put back the original, long names:

\verbatiminput{rename_2_svg.cmd}
\begin{center}
\includegraphics{rename_2_svg.pdf}
\end{center}

[TODO: transform again with longname2english.map]



