\section[sct_topology]{Retaining Topology}

There are cases when one is more interested in the tree's structure than in the
branch lengths, maybe because lengths are irrelevant or just because they are
so short that they obscure the branching order. Consider the following tree,
\filename{vrt1.nw}:

\txtFigure{topol_1_txt.out}

Its structure is not evident, particularly in the upper half. This is because
many branches are short in relation to the depth of the tree, so they are not
well resolved. A better-resolved tree can be obtained by discarding branch
lengths altogether:

\txtCmdOutput{topol_2}

This effectively produces a {\em cladogram}, that is, a tree that represents
ancestry relationships but not amounts of evolutionary change. The inner nodes
are evenly spaced over the depth of the tree, and the leaves are aligned, so
the branching order is more apparent.

Of course, \ascii{} trees have low resolution in the first place, so I'll show
both trees look in \svg. First the original: 

\svgCmdOutput{topol_3}

And now as a cladogram:

\svgCmdOutput{topol_4}

As you can see, even with \svg{}'s much better resolution, it can be useful to
display the tree as a cladogram.

\topology{} has the following options: \code{-b} keeps the branch lengths
(obviously, using this option alone has no effect); \code{-I} discards inner
node labels, and \code{-L} discards leaf labels. An extreme example is the
following, which discards everything {\em but} topology:

\txtFigure{topol_5_txt.cmd}

This produces the following tree, which is still valid Newick:

\txtFigure{topol_5_txt.out}

Let's look at it as a radial tree, for a change:

\svgCmdOutput{topol_6}
