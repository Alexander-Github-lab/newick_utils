\section[sct:distance]{Extracting Distances}

\distance{} prints distances between nodes, in various ways.  By default, it
prints the distance from the root of the tree to each labeled leaf, in \nw{}
order. Let's look at distances in the catarrhinian tree:

\svgFigure{display_3}

\txtCmdOutput{dist_1}

\noindent{}This means that the distance from the root to \id{Gorilla} is
56, etc. The distances are in the same units as the tree's branch lengths --
usually substitutions per site, but this is not specified in the tree itself.
If the tree is a cladogram, the distances are expressed in numbers of
ancestors.  Option \code{-n} shows the labels:

\txtCmdOutput{dist_2}

There are two main parameters to \distance: the {\em method} and the
{\em selection}. The method determines how to compute the distance (from what
node to what node), and the selection determines for which nodes the program is
to compute distances. Let's look at examples.

\subsection{Selection}

In this section we will show the different selection types, using the default
distance method (\ie{}, from the tree's root -- see below). The selection type
is the argument to option \code{-s}.  The nodes appear in the same order as
in the Newick tree, except when they are specified on the command line (see
below).

To illustrate the selection types, we need a tree that has both labeled and
unlabeled leaves and inner nodes. Here it is:

\svgCmdOutput{dist_3}

We will use option \code{-n} to see the node labels.

\subsubsection{All labeled leaves}

The selection consists of all leaves with a label. This is the default, as
leaves will mostly be labeled and we're generally more interested in leaves
than inner nodes.
\txtCmdOutput{dist_4}

\subsubsection{All labeled nodes}

Option \code{-s l}. This takes all labeled nodes into account, whether they are leaves or inner nodes.
\txtCmdOutput{dist_5}

\subsubsection{All leaves}

Option \code{-s f}. Selects all leaves, whether they are
labeled or not.
\txtCmdOutput{dist_6}

\subsubsection{All inner nodes}

Option \code{-s i}. Selects the inner nodes, labeled or not.
\txtCmdOutput{dist_9}

\subsubsection{All nodes}

Option \code{-s a}. All nodes are selected.
\txtCmdOutput{dist_7}

\subsubsection{Command line selection}

The selection consists of the nodes whose labels are passed as arguments on the
command line (after the file name). The distances are printed in the same
order.

\txtCmdOutput{dist_8}

\subsection{Methods}

In this section we will take the default selection and vary the method. The
method is passed as argument to option \code{-m}. I will also use an
\adhoc{} tree to illustrate the methods:

\svgCmdOutput{dist_11}

\noindent{}As explained above, the default selection consists of all labeled
leaves -- in our case, nodes \id{A}, \id{B} and \id{C}. 

\subsubsection{Distance from the tree's root}

This is the default method: for each node in the selection, the program prints
the distance from the tree's root to the node. This was shown above, so I
won't repeat it here.

\subsubsection{Distance from the last common ancestor}
Option \code{-m l}. The program computes the \lca{} of all nodes in the
selection (in our case, node \id{e}), and prints out the distance from
that node to all nodes in the selection.
\txtCmdOutput{dist_10}

\subsubsection{Distance from the parent}
Option \code{-m p}. The program prints the length of each selected node's
parent branch.
\txtCmdOutput{dist_12}

\subsubsection{Matrix}
Option \code{-m m}. Computes the pairwise distances between all nodes in the
selection, and prints it out as a matrix.
\txtCmdOutput{dist_13}

\subsection{Alternative formats}

Option \code{-t} changes the output format. For matrix output, (\code{-m
m}), the matrix is triangular. 
\txtCmdOutput{dist_16}
When labels are printed (option \code{-n}), the diagonal is shown
\txtCmdOutput{dist_14}

For all other formats, the values are printed in a line, separated by
\cap{tab}s.
\txtCmdOutput{dist_15}
